
\documentclass[12pt,leqno]{article}

\usepackage{amsmath,amsfonts,amssymb,amscd,amsthm,amsbsy,upref}


\textheight=8.5truein
\textwidth=6.0truein
\hoffset=-.5truein
\voffset=-.5truein
\numberwithin{equation}{section}
\pagestyle{headings}
\footskip=36pt


\swapnumbers
\newtheorem{thm}{Theorem}[section]

\newtheorem{hthm}[thm]{*Theorem}
\newtheorem{lem}[thm]{Lemma}
\newtheorem{cor}[thm]{Corollary}
\newtheorem{prop}[thm]{Proposition}
\newtheorem{con}[thm]{Conjecture}
\newtheorem*{exer}{Exercise}
\newtheorem{bpe}[thm]{Blank Paper Exercise}
\newtheorem{apex}[thm]{Applications Exercise}
\newtheorem{ques}[thm]{Question}
\newtheorem{scho}[thm]{Scholium}
\newtheorem*{Exthm}{Example Theorem}
\newtheorem*{Thm}{Theorem}
\newtheorem*{Con}{Conjecture}
\newtheorem*{Axiom}{Axiom}



\theoremstyle{definition}
\newtheorem*{Ex}{Example}
\newtheorem*{Def}{Definition}


\newcommand{\lcm}{\operatorname{lcm}}
\newcommand{\ord}{\operatorname{ord}}
\def\pfrac#1#2{{\left(\frac{#1}{#2}\right)}}


\makeindex

\begin{document}




\thispagestyle{plain}
\begin{flushright}
\large{\textbf{Blake Griffith \\}
\textbf{HW 1}: Chapter 2, 1.3, 2.3, 2.4, 3.2,,4.2,4.3,4.4,4.6, 4.10 \& \\
Pre-Lecture Problems: Chapter 2, 2.1, 3.1,4.5, 4.7\\
M373K \\
Due 09/10, 2013 \\}
\end{flushright}

%\maketitle
\markboth{}{} \setcounter{section}{0} \baselineskip=18pt

\setcounter{tocdepth}{4}


%%%%%%%% Write the exercise number inside the brackets.


%%%%%%%%%%%%1.3%%%%%%%%%%%%%%%%%%%%%%%%%%%%%%%%%%%%%%%%%%%%%%%%%%%%%
\begin{exer}[1.3]
Let $\mathbb N $ denote the set $\{1,2,3,\ldots \}$ of natural numbers, and let $s: \mathbb N \rightarrow \mathbb N$ be the shift map, defined by $s(n)=n+1$. Prove that $s$ has no right inverse, but it has infinitely many left inverses. 
\end{exer}

\begin{description}
    \item[right inverse] \hfill \\

    Suppose that $s$ has a right inverse $r$. Then by definition $\forall n \in
    \mathbb N$ we have $s \circ r(n) = n$. However if we take the case $n = 1$ we see:

    \begin{align*}
        s \circ r(1) &= 1 \\
        s(r(1)) &= 1    \\
        r(1) + 1 &= 1   \\
        r(1) &= 0
    \end{align*}

    But this means $s(r(1)) = s(0)$ which is undefined because $s: \mathbb N
    \rightarrow \mathbb N$ and $0 \notin \mathbb N$. So the right inverse
    cannot exist.

    \item[left inverse] \hfill \\

    Suppose $s$ has a left inverse $l$. For any $n \in \mathbb N$ we must have
    $l: \mathbb N \rightarrow \mathbb N$ and

    \begin{align*}
        l \circ s(n) &= n   \\
        l(s(n)) &= n    \\
        l(n + 1) &= n   \\
    \end{align*}

    But the range of $s(n)$ is $\left\{ 2, 3, \dots \right\} $ so our
    requirements on $l \circ s(n)$ only require $l$ be a map from $ \left\{ 2,
    3, \dots \right\} \rightarrow \mathbb N$ so we are free to define $l(1)$ to
    be any of the infinite natural numbers. So there are infinite left inverses
    of $s$.

\end{description}

%%%%%%%%%%%%2.3%%%%%%%%%%%%%%%%%%%%%%%%%%%%%%%%%%%%%%%%%%%%%%%%%%%%%
\begin{exer}[2.3]
Let $x,y,z$  and $w$ be the elements of a group $G$. 
\begin{enumerate}
\item[(a)] Solve for $y$, given that $xyz^{-1}w=1$.
\item[(b)] Suppose that $xyz=1$. Does it follow that $yzx=1$? Does it follow that $yxz=1$?
\end{enumerate}
\end{exer}

\begin{proof}[Proof] \hfill \\

\begin{enumerate}
\item[(a)] \hfill \\

    \begin{align*}
        xyz^{-1}w &= 1  \\
        x^{-1}xyz^{-1}w &= x^{-1}    \\
        yz^{-1}w &= x^{-1}  \\
        yz^{-1}ww^{-1} &= x^{-1}w^{-1}  \\
        yz^{-1} &= x^{-1}w^{-1}  \\
        yz^{-1}z &= x^{-1}w^{-1}z   \\
        y &= x^{-1}w^{-1}z
    \end{align*}

\item[(b)] \hfill \\ 

    For $yzx = yxz = 1$ we would need $xz = zx$ but $x$, $z$ communitivity is
    not implied by the stated conditions. So it does not follow that $yxz = 1$.
    
\end{enumerate}
\end{proof}

%%%%%%%%%%%%2.4%%%%%%%%%%%%%%%%%%%%%%%%%%%%%%%%%%%%%%%%%%%%%%%%%%%%%
\begin{exer}[2.4]
In which of the following cases is $H$ a subgroup of $G$?
\begin{enumerate}
\item[(a)] $G=GL_n(\mathbb C)$ and $H=GL_n(\mathbb R)$.
\item[(b)] $G=\mathbb R^{\times}$ and $H=\{1,-1\}$.
\item[(c)] $G=\mathbb Z^{+}$ and $H$ is the set of positive integers.
\item[(d)] $G=\mathbb R^{\times}$ and $H$ is the set of positive reals.
\item[(e)] $G=GL_2(\mathbb R)$ and $H$ is the set of matrices
$\left ( \begin{array}{cc}
a & 0 \\ 
0 & 0 
\end{array} \right )$ with $a \neq 0$.
\end{enumerate}
\end{exer}

\begin{proof}[Proof]
    \hfill  \\
\begin{enumerate}
\item[(a)] \hfill \\

    Yes. We have closure since $GL_n (\mathbb C) \subset GL_n (\mathbb R)$ and
    $GL_n (\mathbb R)$ is closed under the matrix product. We have the identity
    element since $I \in GL_n (\mathbb R)$. And we have inverses since all
    elements of $GL_n (\mathbb R)$ are invertible by definition and contained
    in $GL_n (\mathbb R)$.

\item[(b)] \hfill \\

    Yes. $H$ is closed under multiplication. The identity element $1 \in H$.
    And finally $1$ and $-1$ are each their own inverse element.

\item[(c)] \hfill \\

    No. The element $644228$ is in $H$ but its inverse, $-644228$, is not in $H$.

\item[(d)] \hfill \\

    Yes. $H$ is closed since for any $a, b \in \mathbb R_{> 0}$ we have $ab >
    0$.  Inverses are in $H$ because if we consider some $a \in \mathbb
    R_{>0}$. Then its inverse is $1/a$ which is also in $\mathbb R$. Finally
    $R_{>0}$ contains the multiplicative identity element $1$.

\item[(e)] \hfill \\

    Yes. We have closure because.

    \begin{align*}
        A = \left ( 
        \begin{array}{cc}
            a & 0 \\
            0 & 0
        \end{array} \right )    \\
        A A = \left ( 
        \begin{array}{cc}
            a^2 & 0 \\
            0 & 0
        \end{array} \right )    \\
        AAA = \left ( 
        \begin{array}{cc}
            a^3 & 0 \\
            0 & 0
        \end{array} \right )    \\
        \dots
    \end{align*}

    And so on... So the $H$ is closed under the matrix product. We have the
    identity element.  Which is the case where $a = 1$.

    \begin{align*}
         \left( 
        \begin{array}{cc}
            1 & 0 \\
            0 & 0
        \end{array} \right)
         \left( 
        \begin{array}{cc}
            a & 0 \\
            0 & 0
        \end{array} \right) 
        = 
         \left( 
        \begin{array}{cc}
            a & 0 \\
            0 & 0
        \end{array} \right)  \\
    \end{align*}

    And

    \begin{align*}
         \left( 
        \begin{array}{cc}
            a & 0 \\
            0 & 0
        \end{array} \right)
         \left( 
        \begin{array}{cc}
            1 & 0 \\
            0 & 0
        \end{array} \right) 
        = 
         \left( 
        \begin{array}{cc}
            a & 0 \\
            0 & 0
        \end{array} \right)  \\
    \end{align*}

    And finally $H$ contains the inverses of all its elements which are of the
    form:

    \begin{align*}
        AA^{-1} = 
         \left( 
        \begin{array}{cc}
            a & 0 \\
            0 & 0
        \end{array} \right)
         \left( 
        \begin{array}{cc}
            1/a & 0 \\
            0 & 0
        \end{array} \right) 
        = 
         \left( 
        \begin{array}{cc}
            1 & 0 \\
            0 & 0
        \end{array} \right)  \\
    \end{align*}
    
\end{enumerate}
\end{proof}



%%%%%%%%%%%%3.2%%%%%%%%%%%%%%%%%%%%%%%%%%%%%%%%%%%%%%%%%%%%%%%%%%%%%
\begin{exer}[3.2]Prove that if $a$ and $b$ are positive integers whose sum is a prime $p$, their greatest common divisor is $1$.

\end{exer}

\begin{proof}[Proof]
\begin{enumerate}

        Let $a + b = c$ where $c$ is prime and define $\gcd (a, b) = d$ where
        $d$ is the biggest number that divides $a$ and $b$. Now since $d$
        divides $a$ and $b$, it must divide $c$. Because we could Rewrite $a +
        b = c$ as

        \begin{align*}
            \underbrace{(d + \dots + d)}_\textrm{a} + \underbrace{(d + \dots +
            d)}_\textrm{b} = \underbrace{(d + \dots + d)}_\textrm{c}
        \end{align*}
        
        However since $c$ is prime its only divisors are $c$ and $1$. So $d$ is
        either $c$ or $1$.  But we can rule out $d = c$ since $d \leq a$ and $d
        \leq b$ and $a + b = c$. Therefore $d = 1$.

\end{enumerate}

\end{proof}



%%%%%%%%%%%%4.2%%%%%%%%%%%%%%%%%%%%%%%%%%%%%%%%%%%%%%%%%%%%%%%%%%%%%
\begin{exer}[4.2]An $n$th root of unity is a complex number $z$ such that $z^n=1$. 
\begin{enumerate}
\item[(a)] Prove that $n$th roots if unity form a cyclic subgroup of $\mathbb C^{\times}$ of order $n$.
\item[(b)] Determine the product of all the $n$th roots of unity. 
\end{enumerate}

\end{exer}

\begin{proof}[Proof]
\begin{enumerate}
\item[(a)] \hfill \\

    If $z$ is an $n$th root of unity. Such that $z^n = 1$. Then we can generate
    then cyclic group $<z>$ with powers of it up to $n - 1$
    \begin{align*}
        <z> = \{z^0, z^1, z^2, \dots z^{n-1} \}
    \end{align*}

    This is closed because any product in $<z>$ can be written $z^i z^j = z^{qn + r}$
    where $q$ is some integer, and $0 < r < n$. By the division algorithm. So we can write
    $z^{i + j} = z^{qn + r} = z^{qn} z^r = z^r$ and $z^r$ is in $<z>$.

    The identity element $1$ is contained in $<z>$. 

    And each elements inverse can be taken as $(z^i)^{-1} = z^{n-i}$.

\item[(b)] \hfill \\

    First lets consider the form of the $n$th root of unity. We rewrite $z$ as
    $z = |z|^n \exp(in\theta)$.  But $|z|^n$ must be one for $z^n = 1$. Now we
    rewrite $\exp(i n \theta)$ using Euler's formula as $z^n = \exp(i n \theta)
    = 1 = \cos (n\theta) + i \sin (n\theta)$. This requires that $\theta = 2
    \pi / n$. So we get the $n$th root of unity in the form $z = \exp{i 2 \pi /
    n}$.

    If we take the product of all roots of unity up to $n$ we see:
    $$
        \prod_{j = 1}^{n} \exp(i 2 \pi /j) = \exp(i2\pi(\sum_{j = 1}^{n} 1/j))    \\
    $$
    But the sum here is a divergent harmonic series... So I don't know what to do.
\end{enumerate}

\end{proof}

%%%%%%%%%%%%4.3%%%%%%%%%%%%%%%%%%%%%%%%%%%%%%%%%%%%%%%%%%%%%%%%%%%%%
\begin{exer}[4.3]Let $a$ and $b$ be elements of a group $G$. Prove that $ab $ and $ba$ have the same order.

\end{exer}

\begin{proof}[Proof]
    Let $ab$ be order $n$, or $(ab)^n = 1$. Then we can rewrite this as:
    
    \begin{align*}
        (ab)^n = (ab)_1 (ab)_2 \dots (ab)_n = 1 \\
    \end{align*}

    We can show this is equivalent to $(ba)^n = 1$ as follows.

    \begin{align*}
        (ab)_1 (ab)_2 \dots (ab)_n &= 1 \\
        a^{-1}(ab)_1 (ab)_2 \dots (ab)_n &= a^{-1}   \\
        (b)_1 (ab)_2 \dots (ab)_n &= a^{-1}  \\
        (b)_1 (ab)_2 \dots (ab)_n a &= a^{-1} a  \\
        (b)_1 (ab)_2 \dots (ab)_n a &= 1 \\
    \end{align*}
    Now shifting the indices.
    \begin{align*}
        (ba)_1 (ba)_2 \dots (ba)_n &= 1 \\
        (ba)^n &= 1  \\
    \end{align*}
    Therefore $ba$ is order $n$.
        
\end{proof}

%%%%%%%%%%%%4.4%%%%%%%%%%%%%%%%%%%%%%%%%%%%%%%%%%%%%%%%%%%%%%%%%%%%%
\begin{exer}[4.4]Describe all groups $G$ that contain no proper subgroup.

\end{exer}

\begin{proof}[Proof]

    For a group to contain no proper subgroup. It either needs to a trivial
    group itself. Or every element in the group can generate the entire group.
    Therefore these groups lacking subgroups must be cyclic, because by our
    definition, a cyclic group can be generated by one element $<x>$.

    So we seek a cyclic group $G$ which has no subgroups. First we consider a
    cyclic $G$ with infinite order. This has infinite subgroups because we can
    take any element $x^i$ and use it to generate a subgroup $<x^i>$ that will
    not contain $x^{i-1}$ therefore $<x^i>$ is proper subgroup, so groups with
    infinite order are ruled out.  \hfill  \\

    For groups with finite order, consider $G$ with order $p$, and some $x^i
    \in G$.  Then $(i \dot n \mod p)$ cannot be zero for some $0 < n < p$.
    Because otherwise, if $(i \dot n \mod p) = 0$ then $x^i$ would generate the
    subgroup $ \{(x^i)^0, (x^i)^1, \dots, (x^i)^n \} $ and since $n < p$ this
    would be a proper subgroup since it contains fewer elements than the parent
    group.  \hfill \\

    This requires that $p$ be prime. Otherwise it would have a divisor $d$ and 
    choosing the $d$th element would yield a subgroup as above. 
    \hfill \\

    So the only groups $G$, without subgroups are cyclic groups with prime orders.
    \hfill \\

\end{proof}

%%%%%%%%%%%%4.6%%%%%%%%%%%%%%%%%%%%%%%%%%%%%%%%%%%%%%%%%%%%%%%%%%%%%
\begin{exer}[4.6] 
\begin{enumerate}
\item[(a)] Let $G$ be a cyclic group of order $6$. How many of its elements generate $G$? Answer the same question for cyclic groups of orders $5$ and $8$.
\item[(b)] Describe the number of elements that generate a cyclic group of arbitrary order $n$.
\end{enumerate}

\end{exer}

\begin{proof}[Proof]
\begin{enumerate}
\item[(a)] \hfill   \\
    A group is generated by its elements if the power of the element is coprime
    with the order of the group.


    To see this consider a group $G$ of order $n$ and an element $x^i \in G$
    where $i$ is not coprime with $n$. Then for some $a <n $ we have $ai = n$
    so we would only generate the subgroup $\{(x^i)^0, (x^i)^1, \dots (x^i)^a
    \}$.

    So for a group of order 6, there are 2 coprimes: 1 and 5. For order 5 we
    have 4 coprimes: 1, 2, 3, 4.  For order 8 we have 4 coprimes: 1, 3, 5, 7.


\item[(b)] \hfill   \\

    The number of elements which generate a cyclic group of order $n$ is equal
    to the number of integers coprime with $n$ and less than $n$.

\end{enumerate}

\end{proof}


%%%%%%%%%%%%4.10%%%%%%%%%%%%%%%%%%%%%%%%%%%%%%%%%%%%%%%%%%%%%%%%%%%%%
\begin{exer}[4.10] Show by an example that the product of elements of finite order in a group need not have finite order. What if the group is abelian? HINT: Think about $2 \times 2$ matrices.

\end{exer}

\begin{proof}[Proof]

    Consider two matrices which are inverses of themselves:
    \begin{align*}
        A = 
         \left( 
        \begin{array}{cc}
            1 & 0 \\
           -1 & 1
        \end{array} \right)
         , A^2 = 
        \left( 
        \begin{array}{cc}
            1 & 0 \\
            0 & 1
        \end{array} \right) \\
        B = 
         \left( 
        \begin{array}{cc}
            1 & 0 \\
            1 & -1
        \end{array} \right)
         , B^2 = 
        \left( 
        \begin{array}{cc}
            1 & 0 \\
            0 & 1
        \end{array} \right)
    \end{align*}

    So $<A>$ and $<B>$ are finite order. But the product $<AB>$ is not.

    \begin{align*}
        AB = 
         \left( 
        \begin{array}{cc}
            1 & 0 \\
           -1 & 1
        \end{array} \right)
        \left( 
        \begin{array}{cc}
            1 & 0 \\
            1 & -1
        \end{array} \right)
        = 
        \left( 
        \begin{array}{cc}
            1 & 0 \\
           -2 & 1
        \end{array} \right) \\
        (AB)^2 =
        \left( 
        \begin{array}{cc}
            1 & 0 \\
           -4 & 1
        \end{array} \right) \\
        (AB)^3 =
        \left( 
        \begin{array}{cc}
            1 & 0 \\
           -8 & 1
        \end{array} \right)
    \end{align*}
    And so on.. So the order of $<AB>$ is infinite. \\

    If we consider two elements of finite order in an Abellian group, it must be true that 
    \begin{align*}
        (ab)^l = \underbrace{(ab)(ab) \dots (ab)}_{l \textrm{ times}} = \underbrace{(a \dots a)}_{l \textrm{ times}} \underbrace{(b \dots b)}_{l \textrm{ times}} = a^l b^l
    \end{align*}

    So if $a$ is order $n$ and $b$ is order $m$, $ab$ is at most order $mn$.

\end{proof}
\section*{Pre-Lecture Problems}
%%%%%%%%%%%%Pre-Lecture Problem 2.1 %%%%%%%%%%%%%%%%%%%%%%%%%%%
\begin{exer}[2.1] Make a multiplication table for the symmetric group $S_3$.
\end{exer}



Using the same notation as on page 42 of the textbook. with rows \(\circ\) columns:
\hfill \\

\begin{center}
\begin{tabular}{|c|c|c|c|c|c|c|}
\hline  &       $1$&        $x$&        $x^2$&      $y$ &       $xy$ &      $x^2y$  \\ 
\hline  $1$&    $1$&        $x$&        $x^2$&      $y$ &       $xy$ &      $x^2y$ \\
    \hline  $x$&    $x$&    $x^2$&  $1$&    $xy$&   $x^2y$& $y$ \\
    \hline  $x^2$&  $x^2$&  $1$&    $x$&    $x^2y$& $y$&    $xy$    \\
    \hline  $y$&    $y$&    $x^2y$& $xy$&   $x$&    $x^2$&  $x$ \\
    \hline  $xy$&   $xy$&   $y$&    $x^2y$& $x^2$&  $1$&    $x^2$   \\
    \hline  $x^2y$& $x^2y$& $xy$&   $y$&    $1$&    $x$&    $1$ \\     
\hline 
\end{tabular}
\end{center}

%%%%%%%%%%%%3.1%%%%%%%%%%%%%%%%%%%%%%%%%%%%%%%%%%%%%%%%%%%%%%%%%%%%%
\begin{exer}[3.1]
Let $a=123$ and $b=321$. Compute $d=gcd(a,b)$ and express $d$ as an integer combination $ra+sb$.
\end{exer}

Applying the Euclidean Algorithm
\begin{align*}
    321 &= 2 \cdot 123 + 75 \\
    123 &= 1 \cdot 75 + 48  \\
    75 &= 1 \cdot 48 + 27   \\
    48 &= 1 \cdot 27 + 21   \\
    27 &= 1 \cdot 21 + 6    \\
    21 &= 4 \cdot 6 + 3 \\
    6 &= 2 \cdot \framebox{3}
\end{align*}

Now we work backwards to find the desired $r$ and $s$.

\begin{alignat*}{2}
    3 &= 21 - 3 \cdot 6 \\
    3 &= 21 - 3(27 - 21) & &= 4 \cdot 21 - 3 \cdot 27 \\
    3 &= -3 \cdot 27 + 4(48 - 27) & &= 4 \cdot 48 - 7 \cdot 27    \\
    3 &= 4 \cdot 48 - 7(75 - 48) & &= -7 \cdot 75 + 11 \cdot 48    \\
    3 &= -7 \cdot 75 + 11(123 - 75) & &= 11 \cdot 123 - 18 \cdot 75   \\
    3 &= 11 \cdot 123 - 18 (321 - 2 \cdot 123)  \\
    3 &= \framebox{47} \cdot 123 - \framebox{18} \cdot 321
\end{alignat*}

%%%%%%%%%%%%4.5%%%%%%%%%%%%%%%%%%%%%%%%%%%%%%%%%%%%%%%%%%%%%%%%%%%%%
\begin{exer}[4.5]
Prove that every subgroup of a cyclic group is cyclic. Do this by working with exponents and use the description of the subgroups of $\mathbb Z^{+}$.
\end{exer}

\begin{proof}[Proof]

    Suppose $G$ is a cyclic group and $H$ is a subgroup of $G$. If $H$ is the
    identity element or equal to $G$ we are done since these are cyclic. If $H$
    is a proper subgroup of $G$, then each element of $H$ must be of the form
    $x^i$ since every element in $G$ has this form. 

    So we can choose the element with lowest positive power, $m$. So $x^m \in
    H$.

    and we choose some other arbitrary element $a = x^n$ of $H$. 

    But by the division algorithm we can write $n = qm + r$ for some integer
    $q$ and $0 \le r < m$. Since $m \le n$.

    So we can write $x^n = x^{qm + r} = (x^m)^q x^r$. But we required that $0
    \le r < m$ and $m$ be the smallest positive power in $H$. So $r = 0$ and
    $x^r = 0$.  So now we have $x^n = (x^m)^q$ and $x^m$ to any power is in $H$
    since it must be closed. So any arbitrary element of $H$ can be written as
    a power of $x$. So $<x> = H$.

\end{proof}

%%%%%%%%%%%%4.7%%%%%%%%%%%%%%%%%%%%%%%%%%%%%%%%%%%%%%%%%%%%%%%%%%%%%
\begin{exer}[4.7]
Let $x$ and $y$ be elements of a group $G$. Assume that each of the elements $x,y$ and $xy$ has order $2$. Prove that the set $H=\{1,x,y,xy\}$ is a subgroup of $G$ and that it has order $4$.
\end{exer}

\begin{proof}[Proof]

    For $H$ to be a subgroup of $G$ it must be closed, contain the identity
    element, and contain each element's inverse. The latter two requirements are
    easily demonstrated:

    \begin{itemize}

        \item identity: The set contains 1. So we have the identity element.

        \item inverses: We are given that each element is order 2. Therefore
            $1^2 = x^2 = y^2 = (xy)^2 = 1$. So each element is its own inverse.

    \end{itemize}

    The requirement of closure can be demonstrated by showing that the Cayley
    table only contains elements which are inside the set. Note that $yx = 1
    \cdot yx = (xy)(xy)(yx) = xyx(yy)x = xy(xx) = xy$.
    \hfill \\

    \begin{center}
    \begin{tabular}{|c|c|c|c|c|}
        \hline  &       $1$&    $x$&    $y$&    $xy$    \\
        \hline  $1$&    $1$&    $x$&    $y$&    $xy$    \\
        \hline  $x$&    $x$&    $1$&    $xy$&   $y$ \\
        \hline  $y$&    $y$&    $xy$&   $y$&    $x$ \\
        \hline  $xy$&   $xy$&   $y$&    $x$&    $1$ \\
        \hline
    \end{tabular}
    \end{center}
    \hfill \\

    We have shown $H$ is a subgroup. Now we can say it is order 4 because it only
    has 4 elements.

\end{proof}


\end{document}
