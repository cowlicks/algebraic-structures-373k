\documentclass[12pt]{article}
\usepackage{amsfonts, amsmath, amsthm}

% a matter of taste
\setlength{\parskip}{1ex}
\setlength{\parindent}{0pt}

\newtheorem*{exer}{Exercise}

% A few simple macros for group theory.
\newcommand{\img}{\text{img }}
\newcommand{\lcm}{\text{lcm }}
\newcommand{\aut}{\text{Aut }}
\newcommand{\cycle}[1]{(\mathbf{#1})}

\begin{document}

\textbf{Homework 6 -- Algebraic Structures} \\

\hrule

% Problem list
\begin{minipage}{.80\linewidth}
    \flushleft
    Ch 7: 7.2, 7.4, 7.5, 7.9, 7.10, 8.3, 8.4, 8.5, M.5, M.6 \\ 
    % ``Pre-lecture problems''
    Pre-lect: 5.1 \\
\end{minipage}
\begin{minipage}{.20\linewidth}
    \flushright
    % whoami
    Blake Griffith
\end{minipage}

% % % % % % % % % % % % % % % % % % % % % % % % % % % % % % % % % % % % 
% Problems
% % % % % % % % % % % % % % % % % % % % % % % % % % % % % % % % % % % % 

\begin{exer}[7.7.2]

    Let $G_1 \subset G_2$ be groups whose orders are divisible by $p$,
    and let $H_1$ be a Sylow p-subgroup of $G_1$. Prove that there is a
    Sylow p-subgroup $H_2$ of $G_2$ such that $H_1 = H_2 \cap G_1$.

\end{exer}

\begin{proof}

    Put a tautology here.

\end{proof}

% % % % % % % % % % % % % % % % % % % % % % % % % % % % % % % % % % % % 

\begin{exer}[7.7.4]

    \begin{enumerate}
        \item Prove that no simple group has order $pq$, where $p$ and
            $q$ are prime.

        \item Prove that no simple group has order $p^2q$, where $p$ and
            $q$ are prime.

    \end{enumerate}

\end{exer}

\begin{proof}

    \begin{enumerate}
        \item Without loss of generality suppose that $p > q$. Then
            there are $s$ sylow p-subpgroups where $s$ must divide $q$.
            So $s$ is either $q$ or $1$. We also know $s$ must be
            congruent to $p \mod 1$. So $s = kp + 1$. But since $p > q$
            there is not $k$ that satisfies $s = q$. So $s$ must be $1$. 

            Since there is only one sylow p-subgroup, it has no other
            conjugate subgroups, by the second sylow theorem. Therefore
            the p-subgroup must be normal. Therefore the group is not
            simple.

        \item Consider the case where $p > q$. Then let $s$ be the
            number of p-subgroups. $s$ must divide $q$, so it is $1$ or
            $q$. And $s$ must satisfy $s =
            pk + 1$ for some integer $k$. Since $p > q$ this is only
            satified when $s = 1$. Making the sylow p-subgroup normal
            and the group non-simple.

            For the case where $p < q$ we let $s$ be the number of sylow
            q-subgroups. $s$ must divide $p^2$ so $s$ is either $1, p$
            or $p^2$. $s$ cannot be $p^2$ since $p < q$ and $s = kq +
            1$. So it is either $1$ or $p^2$. If $s = 1$ the q-subgroup
            is normal and we are done. If we let $s$ be $p^2$ then there
            are $p^2$ subgroups of order $q$ that only intersect
            trivially, so each has $q-1$ unique elements. The combined
            q-subgroups have $(q-1)p^2$ elements. There are only $p^2$
            remaining elements to account for. Then the the sylow
            \textit{p-subgroup} must be order $p^2$, there is only room
            for one of these in the rest of the group. So the p-subgroup
            is normal, and the group is non-simple.

            For the case $p = q$, we have a group of order $p^3$. By the
            third sylow theorem, the number of sylow p-subgroups must
            divide $1$. So there is only one sylow p-subgroup, then the
            p-subgroup is normal, and the group is non-simple.

    \end{enumerate}

\end{proof}

% % % % % % % % % % % % % % % % % % % % % % % % % % % % % % % % % % % % 

\begin{exer}[7.7.5]

    Find Sylow 2-subgroups of $D_{10}$.

\end{exer}

\begin{proof}

    Recall that the $D_{10}$ is order $20$, and we choose the
    representation $r$ a rotation of $36^\circ$ and $l$ is a reflection
    across the vertical axis of symmetry, so $r^{10} = l^2 = 1$. Since
    the groups order is $20 = 2^2 5$ the group has either $5$ or $1$
    sylow 2-subgroups.

    If there were $5$ sylow 2-subgroups, this would account for $3
    \times 5 = 15$ elements which could have at most an order of 4. But
    there are more than $5$ elements with order greater than 4: $r, r^2,
    r^3, r^4, r^6, r^7,$ etc. So having 5 sylow 2-subgroups is
    impossible.

    So there is only 1 sylow 2-subgroup and the only feasable choice is
    that generated by two order 2 element $r^5, l$ which is $\{1, r^5,
    l, r^5l\}$. Also note that $r^5 l = l r^5$.

    

\end{proof}

% % % % % % % % % % % % % % % % % % % % % % % % % % % % % % % % % % % % 

\begin{exer}[7.7.9]

    Classify groups of order \textbf{(1)} 33 \textbf{(2)} 18.

\end{exer}

\begin{proof}

    \begin{enumerate}
        \item Note that $33 = 11 \times 3$. Let the number of sylow
            11-subgroups be $s$. Then by the third sylow theorem $s$
            must divide 3, and $s = k11 + 1$. The only choice of $s$
            that works here is $1$. So there is only 1 sylow
            11-subgroup. Now let the number of sylow 3-subgroups be $r$.
            Thene $r$ must divide 11 and $r = k3 + 1$. The only choice
            here is $r = 1$. So there is 1 sylow 3-subgroup.

            Any group of order 33 must contain bothe of the cyclic
            subgroups. The product of the sylow groups must be in the
            group. And since the product of the sylow groups is order
            33, it must be equal to the group. 

            So all groups of order 33 are isomorphic to $C_{3} C_{11}$

        \item Note that $18 = 3^2 \times 2$.

    \end{enumerate}<++>

\end{proof}

% % % % % % % % % % % % % % % % % % % % % % % % % % % % % % % % % % % % 

\begin{exer}[7.7.10]

    Prove that the only simple groups of order $<$ 60 are the groups of
    prime order.

\end{exer}

\begin{proof}

    First we list all the numbers less than 60 that are not prime 

    \begin{verbatim}
     4   6   8 9 10    12    14 15 16    18    20 21 22    24 25
     26 27 28    30    32 33 34 35 36    38 39 40    42    44 45 46
     48 49 50 51 52    54 55 56 57 58
     \end{verbatim}

     Next we note that from previous problems that groups of order $pq$
     are not simple.

     \begin{verbatim}
             8         12          16    18    20          24   
        27 28    30    32          36          40    42    44      
           50    52    54    56      
     \end{verbatim}

     Next recall that groups of order $p^2q$ cannot be simple.
        
     \begin{verbatim}
                                   16                      24   
                 30    32          36          40    42            
                       54    56      
     \end{verbatim}

     Now recall that every group that is of order $p^n$ has a
     non-trivial center, and therefore a normal subgroup. Eliminating
     these gives:

     \begin{verbatim}
                                                           24   
                 30                36          40    42            
                       54    56      
     \end{verbatim}

\end{proof}

% % % % % % % % % % % % % % % % % % % % % % % % % % % % % % % % % % % % 

\begin{exer}[7.8.3]

    Determine the class equations of the groups of order 12.

\end{exer}

\begin{proof}

\end{proof}

% % % % % % % % % % % % % % % % % % % % % % % % % % % % % % % % % % % % 

\begin{exer}[7.8.4]

    Prove that a group of order $n = 2p$, where $p$ is a prime, is
    either cyclic or dihedral.

\end{exer}

\begin{proof}

\end{proof}

% % % % % % % % % % % % % % % % % % % % % % % % % % % % % % % % % % % % 

\begin{exer}[7.8.5]

    Let $G$ be a nonabelian group of order 28 whose sylow 2 subgroups
    are cyclic.

    \begin{enumerate}
        \item Determine the numbers of sylow 2 subgroups and sylow 7
            subgroups.

        \item Prove that there is at most one isomorphism class of such
            groups.

        \item Determine the numbers of elements of each order, and the
            class equation of $G$.
    \end{enumerate}

\end{exer}

\begin{proof}

\end{proof}

% % % % % % % % % % % % % % % % % % % % % % % % % % % % % % % % % % % % 

\begin{exer}[7.M.5]

    Let $H$ and $N$ be subgroups of a group $G$, and assume that $N$ is
    a normal subgroup.

    \begin{enumerate}
        \item Determine the kernels of the restrictions of the canonical
            homomorphism $\pi : G \rightarrow G/N$ ot the subgroups $H$
            and $HN$. 

        \item Applying First Isomorphism Theorem to these restrictions,
            prove the \textit{Second Isomorphism Theorem}: $H/(H \cap
            N)$ is isomorphic to $(HN)/N$. 
    \end{enumerate}

\end{exer}

\begin{proof}

\end{proof}

% % % % % % % % % % % % % % % % % % % % % % % % % % % % % % % % % % % % 

\begin{exer}[7.M.6]

    Let $H$ and $N$ be normal subgroups of a group $G$ such that $H
    \supset N$. Let $\overline{H} = H/N$ and $\overline{G} = G/N$.

    \begin{enumerate}
        \item Prove that $\overline{H}$ is a normal subgroup of
            $\overline{G}$. 
        \item Use the composed homomorphism $G \rightarrow \overline{G}
            \rightarrow \overline{G}/\overline{H}$ to prove the
            \textit{Third Isomorphism Theorem}: $G/H$ is isomorphic to
            $\overline{G} / \overline{H}$.
            
    \end{enumerate}

\end{exer}

\begin{proof}

\end{proof}

\end{document}
