\documentclass[12pt]{article}
\usepackage{amsfonts, amsmath, amsthm}

\setlength{\parskip}{1ex}
\setlength{\parindent}{0pt}

\newtheorem*{exer}{Exercise}

\newcommand{\img}{\text{img }}
\newcommand{\lcm}{\text{lcm }}
\newcommand{\cycle}[1]{(\mathbf{#1})}

\begin{document}

\textbf{Homework 3 -- Algebraic Structures} \\

\hrule

\begin{minipage}{.80\linewidth}
    \flushleft
    Ch 2: 8.8, 8.9, 8.10, 10.3, 10.4, 10.5, 11.2, 11.5, 11.6, 11.9,
    12.4, 12.5, M.3  \& \\ 
    Pre-lect: Chapter 2, 9.1, 10.1, 11.1, 12.1\\
\end{minipage}
\begin{minipage}{.20\linewidth}
    \flushright
    Blake Griffith
\end{minipage}

\begin{exer}[8.8]
Let $G$ be a group of order $25$. Prove that $G$ has at least one
subgroup of order $5$, and that if it contains only one subgroup of
order $5$, then it is a cyclic group.

\end{exer}

\begin{proof}
    Suppose $G$ is cyclic with some generator $x$. Then $x^5$ generates
    a subgroup of order 5. $<x^5> = \{1, x^5, x^{10}, x^{15}, x^{20}\}$. 

    Suppose $G$ is not cyclic. Then there is no element that generates
    all of $G$. So taking some element $x$ in $G$ (where $x$ is not the
    identity). This element must generate a subgroup smaller than $G$. By
    Lagrange's theorem, this subgroup can only be order $5$.

    So $G$ must contain at least one subgroup of order 5.

    Now if $G$ only contains one proper subgroup, $H$, and $H$ is order
    5. Then any element $x$ in $G$ but not in $H$ must generate $G$
    otherwise it would generate a proper subgroup. So since $<x> = G$,
    $G$ is cyclic.

\end{proof}

% % % % % % % % % % % % % % % % % % % % % % %

\begin{exer}[8.9]
Let $G$ be a finite group. Under what circumstances is the map $\phi:G
\rightarrow G$ defined by $\phi(x)=x^2$ an automorphism of $G$?
\end{exer}

\begin{proof}

    First every element other than the identity must have odd order.
    Otherwise, if we had some element $x$ with even order $2N$ so that
    $x^{2N} = 1$ then $\phi(x^N) = 1 = \phi(1)$. So $\phi$ would not be
    injective. Therefore $\phi$ could not be an automorphism since
    $|\img \phi | < |G|$ so $\img \phi \neq G$.

    Now since each element has odd order, no element can be its own
    inverse, since if some element $x$ was its own inverse it would
    generate the even ordered subgroup $<x> = \{1, x\}$. So for each
    (non identity) element in our group there is a distinct
    corresponding inverse, so the elements must come in pairs. So there
    is an even number of non identity elements.

    So there is some even number of non identity elements plus the
    identity, so the order of the group must be odd.

    So we can say $\phi$ can be an automorphism on a group $G$ if $G$
    has odd order and every non identity element has odd order.

\end{proof}

% % % % % % % % % % % % % % % % % % % % % % %

\begin{exer}[8.10]
Prove that every subgroup of index $2$ is a normal subgroup, and show by
an example that a subgroup of index $3$ need not be normal.
\end{exer}

\begin{proof}

    First, there are the same number of right cosets as left cosets.
    This is because there is a bijection between left and right cosets. 
    Let this bijection be $\phi$  such that $\phi(gH) = Hg^{-1}$.

    $\phi$ is well defined, if we have some $a$ and $b$ such that $aH =
    bH$ then $a^{-1} b \in H$. So $\phi(aH) = Ha^{-1}$ and $\phi(bH) =
    Hb^{-1}$. Now $Ha^{-1} \subseteq Hb^{-1}$ since for any $h_1$ in $H$
    there is some $h_2$ such that $h_1 a^{-1}b = h_2 \implies h_1 a^{-1}
    = h_2 b^{-1}$. Likewise $Ha^{-1} \supseteq Hb^{-1}$. So $\phi$ is
    well defined.

    $\phi$ is injective. Suppose there exists some $xH$ and $yH$ such
    that $\phi(xH) = \phi(yH) \implies Hx^{-1} = Hy^{-1}$ then $x^{-1} =
    hy^{-1}$ for some $h$ in $H$ then $yx^{-1} \in H$. Then $xH
    \subseteq yH$ because for any $h_1 \in H$ there is a $h_2 \in H$
    such that $y^{-1}xh_1 = h_2 \implies xh_1 = yh_2$. Likewise $xH
    \supseteq yH$. So $xH = yH$. So 

    $\phi$ is surjective. Let $Hx$ be some right coset, then it's
    preimage is $x^{-1}H$ since $\phi(x^{-1}H) = Hx$.

    This establishes that there are the same number of left cosets as
    right cosets. 

    Call this group with index 2 $H$. Then of the 2 left cosets
    that partition the group, one of them must be $H$ since for any $h$
    in $H$ $hH = H$. Likewise $H = Hh$. 

    Let the other left coset be $g_1 H$ for any $g_1 \in G - H$. And let
    the corresponding right coset be $H g_2$ for any $g_2 \in G - H$.
    Since these cosets partition the group, we have $G = H \cup g_1 H =
    H \cup Hg_2$ or $g_1 H = H g_2$. If we take $g_1 = g_2 = g$ then $gH
    = Hg$. So for any $h_1$ in $H$, $g$ in $G$, there exists a $h_2$
    such that $gh_1 = h_2 g \implies gh_1 g^{-1} = h_2$. So $H$ is a
    normal subgroup.
    
    For an example of a subgroup that is index 3 and not normal. See the
    subgroup $<y>$ in $S_3$. Its left cosets are $\{1, y\}$, $\{x,
    xy\}$, and $\{x^2, x^2y\}$ so its index is 3. And it is not normal
    in $S_3$ since $xyx^{-1} = xyx^2 = xy$, which is not in $<y>$.

\end{proof}

% % % % % % % % % % % % % % % % % % % % % % %

\begin{exer}[10.3]
Let $G$ and $G'$ be cyclic groups of order $12$ and $6$, generated by
elements $x$ and $y$, respectively, and let $\phi: G \rightarrow G'$ be
the map defined by $\phi(x^i)=y^i$. Exhibit the correspondence  referred
in the Correspondence Theorem explicitly.
\end{exer}

\begin{proof}

    $\phi$ is surjective, if we take $y^i$ in $G'$ this is mapped by
    $x^i$ in $G$.

    $\ker \phi$ is all $x^i \in G$ such that $\phi(x^i)= 1 = y^6$. This
    is given by $\ker \phi = \{1, x^6\}$.

    First we find the groups in $G$ that contain $\ker \phi$. $G$ and
    $K$ are obvious the others are generated by $x^n$ where $n$ divides
    $6$.  So $G_3 = \{1, x^3, x^6, x^9\}$ and $G_2 = \{1, x^2, x^4, x^6,
        x^8, x^{10}\}$. There are four of these. 

    Now we find all the subgroups of $G'$. These are given by $\phi(K)$,
    $\phi(G)$, $\phi(G_3)$, and $\phi(G_2)$. So they are $\{1\}$, $G'$,
    $G'_2 = \{1, y^2, y^4\}$, and $G'_3 = \{1, x^3\}$.

    So there are 4 subgroups of $G$ which contain $K$ and 4 subgroups of
    $G'$. So there is a bijection between the two since they are the
    same size.

\end{proof}

% % % % % % % % % % % % % % % % % % % % % % % %

\begin{exer}[10.4]
With the notation of the Correspondence Theorem, let $H$ and $H'$ be
corresponding groups. Prove that $[G:H]=[G':H']$.
\end{exer}

\begin{proof}

    \begin{align*}
        [G:H] &= \frac{|g|}{|H|} &\text{ By counting theorem} \\
        &= \frac{|K| [G:K]}{|H|} &\text{ By counting theorem} \\
        &= \frac{|K| |\img \phi}{|H|} &\text{ Left cosets of $\ker \phi$
        are in bijection with $\img \phi$} \\
        &= \frac{|K||G'|}{H|} &\text{ $\phi$ is surjective so $\img \phi
        = G'$} \\
        &= \frac{|K||G'|}{|K||H'|} &\text{ By correspondence theorem} \\
        &= \frac{|G'|}{|H'|} &\text{ Cancellation}\\
        &= [G':H'] &\text{By counting theorem}
    \end{align*}

    So $[G:H] = [G':H']$.

\end{proof}
% % % % % % % % % % % % % % % % % % % % % % % %

\begin{exer}[10.5]
With the reference to the homomorphism $S_4 \rightarrow S_3$ described
in Example 2.5.13, determine the six subgroups of $S_4$ that contain
$K$.
\end{exer}

\begin{proof}

    From the example we know $K = \{ 1, \cycle{12} \cycle{34},
    \cycle{13}\cycle{24}, \cycle{14}\cycle{23}\}$ so $|K| = 4$. From
    the correspondence theorem we know that for a subgroup $H$ in $S_4$
    and its corresponding subgroup $H'$ in $S_3$, $|H| = |K||H'|$. So
    the corresponding subgroups in $S_4$ are order 4, 8$\times$3, 12 and
    24. Obviously the order 4 and 24 subgroups are $K$ and $S_4$,
    respectively. We know $A_4$ in $S_3$ is order 12 and contains all
    the even permutations and all the permutations in $K$ are even, so
    $A_4$ must correspond with $<x>$ in $S_3$.
    
    Now we must find the subgroups in $S_4$ that correspond to $<y>,
    <xy>, <x^2y>$ in $S_3$. First consider $<y>$, by the definition of
    the homomorphism we can see \[ y = \cycle{12} = \{\Pi_1, \Pi_2\}
        \implies \Pi_1 : \{1,2\}\cup\{3,4\} \leadsto \Pi_2:
    \{1,3\}\cup\{2,4\} \] So the corresponding cycle in $S_4$ must be
    $\cycle{23}$. This cycle and $K$ gives the subgroup:  $K +
    \{\cycle{23}, \cycle{1243}, \cycle{1342}, \cycle{14} \}$.

    Now the same process for $x^2y$ yields a corresponding subgroup of $K +
    \{\cycle{34}, \cycle{1324}, \cycle{12}, \cycle{1423}\}$.

    With $xy$ this yields a corresponding subgroup of $K + \{\cycle{24},
    \cycle{1234}, \cycle{13}, \cycle{1432}\}$.

\end{proof}

% % % % % % % % % % % % % % % % % % % % % % % % %

\begin{exer}[11.2]
What does Proposition 2.11.4 tell us when, with the usual notation for
the symmetric group $S_3$, $K$ and $H$ are the subgroups $\langle y
\rangle$ and $\langle x \rangle$?
\end{exer}

\begin{proof}
    We are treating $<x>$ as $H$ and $<y>$ as $K$.
    \begin{itemize}

        \item $<x> \cap <y> = {1}$ so $f$ is injective.

        \item $yx = x^2 y \neq xy$ so $f$ is not a group homomorphism
            from $<x> \times <y>$ to $S_3$.

        \item To check if $<x>$ is a normal subgroup we check if its
            index is 2 (See exercise 8.10). Obviously
            the cosets $1<x>$, $x<x>$, and~$x^2<x>$ are $<x>$. So we
            compute the nontrivial cases.
            \begin{align*}
                y<x> &= \{y, yx, yx^2\} &= \{y, x^2y, xy\} \\
                xy<x> &= \{xy, xyx, xyx^2 \} &= \{y, x^2y, xy\} \\
                x^2y<x> &= \{x^2y, x^2yx, x^2yx^2\} &= \{y, x^2y, xy\}
            \end{align*}

            So $[S_3 : <x>] = 2$, so $<x>$ is normal in $S_3$. So
            $<x><y>$ is a subgroup of $S_3$.

        \item $<y>$ is not normal in $S_3$ since $xyx^{-1} = xyx^2 = xy$
            which is not in $<y>$. So $f$ is not an isomorphism from
            $<x> \times <y>$ to $S_3$.

    \end{itemize}

\end{proof}

% % % % % % % % % % % % % % % % % % % % % % % % % %


\begin{exer}[11.5]
Let $G_1$ and $G_2$ be groups and let $Z_i$ be the center of $G_i$.
Prove that the center of the product group $G_1 \times G_2$ is $Z_1
\times Z_2$.
\end{exer}

\begin{proof}

    If $Z_1 \times Z_2$ is in the center $C$ of $G_1 \times G_2$ then 
    for all $g$ in $G_1 \times G_2$ and $z$ in $Z_1 \times Z_2$, $zg =
    gz$.

    We can rewrite $g$ and $z$ using the definition of a product group.
    $g = (g_1, g_2)$ for some $g_1$ in $G_1$ and $g_2$ in $G_2$ likewise
    $z = (z_1, z_2)$ for some $z_1$ in $Z_1$ and $z_2$ in $Z_2$. So
    \begin{align*}
        zg &= (z_1, z_2)(g_1, g_2) \\
           &= (z_1 g_1, z_2 g_2) &\text{ by definition of product group}
           \\
           &= (g_1 z_1, g_2 z_2) &\text{ since $z_1$ and $z_2$ are in the
           center} \\
           &= (g_1, g_2)(z_1, z_2) &\text{ definition of product group}
           \\
           &= gz
    \end{align*}
    So any element in $Z_1 \times Z_2$ commutes with any element of $G_1
    \times G_2$. So $Z_1 \times Z_2$ is in the center of $G_1 \times
    G_2$, i.e. $Z_1 \times Z_2 \subseteq C$ 

    Any element in the center must commute with any element of the
    group.  So we take some arbitrary element $c$ in the center $C$.  By
    definition of the center we must have $cg = gc$ for any $g$  in $G$.
    We can rewrite $c$ and $g$ using the definition of product groups.
    $c = (c_1, c_2)$ for some $c_1, c_2$ in $C_1$ and $C_2$. And $g =
    (g_1, g_2)$ for some $g_1$, $g_2$ in $G_1$ and $G_2$.
    
    \[
        cg = (c_1, c_2)(g_1, g_2) = (c_1g_1, c_2g_2) = (g_1c_1, g_2c_2)
        = (g_1, g_2)(c_1, c_2) = gc
    \]

    So any element in $C_1$ commutes with any element of $G_1$, likewise
    for $C_2$. So $C_1$ must in the center $Z_1$ of $G_1$ likewise $C_2$
    is in the center $Z_2$ of $G_2$. So $C \subseteq Z_1 \times Z_2$.

\end{proof}

% % % % % % % % % % % % % % % % % % % % % % % % % %

\begin{exer}[11.6]
Let $G$ be a group that contains normal subgroups of order $3$ and $5$,
respectively. Prove that $G$ contains an element of order $15$.
\end{exer}

\begin{proof}

    We denote the order $3$ subgroup as $H$ and the order $5$ group as
    $K$. Note that $H$ and $K$ must be cyclic since they are prime
    order.

    We apply proposition 2.11.4.d to show that there is a isomorphism
    between a subgroup $S$ of $G$ and the product group $H \times K$;
    which is a cyclic group of order $15$ by proposition 2.11.3 and
    therefore has an element of order $15$.

    We take $HK = S$. $S$ is guaranteed to be a group by proposition
    2.11.4.C, since $H$ and $K$ are normal. Since $H$, and $K$ are
    normal in $G$ they are normal in its subgroups, i.e. $S$.

    We check $H \cap K = \{1\}$. Suppose $H$ and $K$ shared some
    non identity element $x$. Since $H$ and $K$ are prime order each of
    their elements must generate the whole group. So $x$ would have to
    generate $H$ and $K$, but this is impossible. So $x$ does not exist
    and $H \cap K = \{1\}$.

    So $H \cap K = \{1\}$, $HK = S$, and $H, K$ are normal in
    $G$. Therefore there is an isomorphism between $H \times K$ and $S$.

    Since $H \times K$ is the product of two cyclic groups of order $3$
    and $5$ its order is $3 \times 5 = 15$. Also $H \times K$ is cyclic
    because its factors have coprime order. So since $H \times K$ is
    cyclic and order $15$ its generator is order $15$. Since $H \times
    K$ is isomorphic to a subgroup of $G$ there must be an element in
    $G$ that is order $15$.

\end{proof}

% % % % % % % % % % % % % % % % % % % % % % % % % % % % %

\begin{exer}[11.9]
Let $H$ and $K$ be subgroups of a group $G$. Prove that the product set
$HK$ is a subgroup of $G$ if and only if $HK=KH$.
\end{exer}

\begin{proof}

    If $HK$ is a subgroup of $G$ it must be closed. So that $HKHK = HK$.
    This is true only if $HK = KH$ since it implies $HKHK = HHKK = HK$.

    If $HK = KH$ then $H$ is a group because it is
    \begin{itemize}
        \item Has inverses. If we take some element $a$ in $HK$. Then we
            can factor it to $hk$ where $h$ is in $H$ and $k$ is in $K$.
            So $a^{-1} = (a)^{-1} = (hk)^{-1} = k^{-1} h^{-1}$.
            $h^{-1}k^{-1}$ is in $HK$ since $h^{-1}$ is in $H$ and
            $k^{-1}$ is in $K$.

        \item Has identity (1). 1 is in $H$ and $K$ so $1 = 1 \cdot 1$
            is in $HK$.

        \item Has closure.  $HKHK = HHKK$ since $HK = KH$. So $HKHK =
            HHKK = HK$.

    \end{itemize}

\end{proof}

% % % % % % % % % % % % % % % % % % % % % % % % % % % % %

\begin{exer}[12.4]
Let $H=\{\pm1, \pm i\}$ be the subgroup of $G=\mathbb C^\times$ of
fourth roots of unity. Describe the cosets of $H$ in $G$ explicitly. Is
$G/H$ isomorphic to $G$?
\end{exer}

\begin{proof}

    For some element $a + ib$ of $\mathbb{C}^\times$. The coset is
    $(a + ib) = \{a + ib, -a - ib, -b + ia, b - ia\}$. 

    We can see that $G/H$ is not isomorphic to $G$ if we note $(1 + i)H
    = (-1 - i)H = \{1 + i, -1 - i, -1 + i, 1 - i\}$. So a map $\pi$
    between $G/H$ and $G$ could not be isomorphic since $\pi(1 + i) =
    \pi(-1 - i)$.

\end{proof}

% % % % % % % % % % % % % % % % % % % % % % % % % % % % % %

\begin{exer}[12.5]
Let $G$ be the group of upper triangular matrices $\left
[\begin{array}{cc} a & b  \\ 0 & d \end{array} \right ]$, with $a$ and
$d$ different from $0$. For each of the following subsets determine
whether or not $S$ is a subgroup, and whether or not $S$ is a normal
subgroup. If $S$ is a normal subgroup, identify the quotient group
$G/S$.
\begin{enumerate}
\item $S$ is the subset defined by $b=0$.
\item $S$ is the subset defined by $d=1$.
\item $S$ is the subset defined by $a=d$.
\end{enumerate}
\end{exer}

\begin{proof}

    Note that for some matrix $g$ in $G$. Where $
    g = \left[
        \begin{array}{cc}
            a & b \\
            0 & d
        \end{array}
    \right]$, It's inverse is $g^{-1} = 
    \left[
        \begin{array}{cc}
            1/a & -b/(da) \\
            0 & 1/d
        \end{array}
    \right]$.

    \begin{enumerate}

        \item $S$ is a subgroup. To see this consider some element $s$
            in $S$ where 
            $ s= 
                \left[
                    \begin{array}{cc}
                        a & 0 \\
                        0 & d
                    \end{array}
                \right]$. 
            $S$ contains inverses, $s^{-1} =
                \left[
                    \begin{array}{cc}
                        1/a & 0 \\
                        0   & 1/d
                    \end{array}
                \right]$. 
            $S$ contains the identity matrix. $S$ is closed since
            $ ss = 
                \left[
                    \begin{array}{cc}
                        a^2 & 0 \\
                        0 & d^2
                    \end{array}
                \right] \in S$.

            However $S$ is not normal, consider some matrix $g$ in $G$ 
            $g = \left[
            \begin{array}{cc}
                a' & b' \\
                0 & d'
            \end{array}
            \right]$
            Then $gsg^{-1} = \left[
            \begin{array}{cc}
                a & b'(d - a)/d' \\
                0 & d
            \end{array}
            \right]$. Which is not in $S$.

        \item $S$ is a subgroup. To see this consider some element $s$
            in $S$ where 
            $ s= 
                \left[
                    \begin{array}{cc}
                        a & b \\
                        0 & 1
                    \end{array}
                \right]$. 
            $S$ contains inverses, $s^{-1} =
                \left[
                    \begin{array}{cc}
                        1/a & -b/a \\
                        0   & 1
                    \end{array}
                \right]$. 
            $S$ contains the identity matrix. $S$ is closed since
            $ ss = 
                \left[
                    \begin{array}{cc}
                        a^2 & b(a + 1) \\
                        0 & 1
                    \end{array}
                \right] \in S$.

            $S$ is normal in $G$. Consider some matrix $g$ in $G$ 
            $g = \left[
            \begin{array}{cc}
                a' & b' \\
                0 & d'
            \end{array}
            \right]$
            Then $gsg^{-1} = \left[
            \begin{array}{cc}
                a & b'(1 - a)/d' \\
                0 & 1
            \end{array}
            \right]$. Which is in $S$.

            $S$ is the kernel of the homomorphism $\phi G \rightarrow G'$
            where $\phi(X) = X\left[
            \begin{array}{cc}
                1/x_{1,1} & x_{1,2}/x_{1,1} \\
                0 & 1
            \end{array}
            \right]$. Where $X_{i,j}$ is the element in $i$th row and
            $j$th column of $X$. So by the first isomorphism theorem,
            $G/S$ is isomorphic to $\phi(G)$.

        \item $S$ is a subgroup. To see this consider some element $s$
            in $S$ where 
            $ s= 
                \left[
                    \begin{array}{cc}
                        a & b \\
                        0 & a
                    \end{array}
                \right]$. 
            $S$ contains inverses, $s^{-1} =
                \left[
                    \begin{array}{cc}
                        1/a & -b/a^2 \\
                        0   & 1/a
                    \end{array}
                \right]$. 
            $S$ contains the identity matrix. $S$ is closed since
            $ ss = 
                \left[
                    \begin{array}{cc}
                        a^2 & 2ab \\
                        0 & a^2
                    \end{array}
                \right] \in S$.

            $S$ is normal in $G$. Consider some matrix $g$ in $G$ 
            $g = \left[
            \begin{array}{cc}
                a' & b' \\
                0 & d'
            \end{array}
            \right]$
            Then $gsg^{-1} = \left[
            \begin{array}{cc}
                a & (-a'b + b'a + a')/d \\
                0 & a
            \end{array}
            \right]$. Which is in $S$.

            $S$ is the kernel of the homomorphism $\phi G \rightarrow G'$
            where $\phi(X) = X\left[
            \begin{array}{cc}
                1/x_{1,1} & - x_{1,2}/(x_{1,1} x_{2,2} \\
                0 & 1/x_{2,2}
            \end{array}
            \right]$. Where $X_{i,j}$ is the element in $i$th row and
            $j$th column of $X$. So by the first isomorphism theorem,
            $G/S$ is isomorphic to $\phi(G)$.

    \end{enumerate}
\end{proof}

% % % % % % % % % % % % % % % % % % % % % % % % % % % % % % % % %

\begin{exer}[M.3]
Classify groups of order $6$ by analyzing the following cases:
\begin{enumerate}
\item $G$ contains an element of order $6$.
\item $G$ contains an element of order $3$ but none of order $6$.
\item All elements of $G$ have order 1 or $2$.
\end{enumerate}
\end{exer}

\begin{proof}
    \begin{enumerate}

        \item $G$ is isomorphic to a cyclic group of order $6$. This is
            clear because the element of order $6$ must generate the
            group since it is the same order as the group.

        \item So we know this group has an element of order $3$, call it
            $x$. So $<x> = \{1, x, x^2\}$. This leaves $3$ more elements
            in $G$. There cannot be anymore elements of order $3$
            because these come in pairs, and would account for the
            remaining $3$ elements (since there cannot be another
            element of order 1). So the remaining elements must all be
            of order $2$. If we call these $y, xy$, and $x^2y$ we
            clearly have $S_3$.

        \item This is impossible. To see this suppose $G$ exists.

            The only element in $G$ of order $1$ is the identity. So all
            remaining elements would have to be order $2$. So we can
            write $G$ as the identity and $5$ distinct elements.  
            
            \[ 
                G = \{1, a, b, c, d, e\}, \quad a^2 = b^2 = c^2 = d^2 =
                e^2 = 1
            \]

            Note that these elements are abellian $ab = (ab)^{-1} =
            b^{-1}a^{-1} = ba$.  We can define the products of this
            group, consider $ab$. It cannot be $1, a,$ or $b$, since $1$
            would violate the uniqueness of the inverse, or $a, b$ since
            it would violate the uniqueness of the identity. So without
            loss of generality we choose $ab = c$. Likewise for $ab$ we
            choose $ad = e$, since it cannot be $1, a, b, c, d$. Now if
            we consider $bd$. It obviously cannot be $1, b$, or $d$.
            Choosing $bd = a$ leads to $a = c$. Choosing $bd = c$ leads
            to $a = d$. Choosing $ab = e$ leads to $a = d$. So $bd$
            cannot violate the group, this violates closure of the group
            so we have a contradiction. 

    \end{enumerate}

\end{proof}

% % % % % % % % % % % % % % % % % % % % % % % % % % % % % % % % % %
% % % % % % % % % % % % % % % % % % % % % % % % % % % % % % % % % %
\section*{Pre-Lecture Problems}

\begin{exer}[9.1]
For which integers $n$ does $2$ have a multiplicative inverse in
$\mathbb Z/ \mathbb Z n$?
\end{exer}

\begin{proof}

    Let $a$ be such that $2a = a2 = 1$. For a given $n$ we know that $n
    = 1 = 2a \implies a = n/2$. $a$ must be an integer so $n$ must be
    even.

\end{proof}

% % % % % % % % % % % % % % % % % % % % % % % % % % % % % % % % % %

\begin{exer}[10.1]
Describe how to tell from the cycle decomposition whether a permutation
is odd or even.
\end{exer}

\begin{proof}

    The parity of a cycle decomposition can be determined from the
    number of 2-cycles, or transpositions, it can be written as. If a
    cycle can decomposed into an even number of 2-cycles, it is even. If
    it can be decomposed into an odd number of 2-cycles then it is odd.

\end{proof}


% % % % % % % % % % % % % % % % % % % % % % % % % % % % % % % % % % % %

\begin{exer}[11.1]
Let $x$ be an element of order $r$ of a group $G$, and let $y$ be an
element of $G'$ of order $s$. What is the order of $(x,y)$ in the
product group $G \times G'$?

\end{exer}

\begin{proof}

     The order of $(x, y)$ is the least common multiple of $r$ and $s$,
     $\lcm(r,s)$.

     Let $n = \lcm(r, s)$. Then $(x,y)^n = (x^n, y^n) = (1, 1)$ since
     $n$ is the smallest integer that $r$ and $s$ divide by definition
     of the $\lcm$. 

\end{proof}

% % % % % % % % % % % % % % % % % % % % % % % % % % % % % % % % % % % %

\begin{exer}[12.1]
Show that if a subgroup $H$ of a group $G$ is not normal, there are left
cosets $aH$ and $bH$ whose product is not a coset. 
\end{exer}

\begin{proof}

$S_3$'s subgroups $<x>$ and $<y>$ are not normal.Consider the cosets $x<y>$
and $y<x>$. Their product is $y<x>x<y> = \{x^2y, x^2, xy, x, y, 1\}$.
Since this product is order 6. It cannot be produced by a coset of a proper
subgroup of $S_3$. Therefore $y<x>x<y>$ is not a coset.
    \begin{align*}
    \end{align*}

\end{proof}

% % % % % % % % % % % % % % % % % % % % % % % % % % % % % % % % % % % %



\end{document}
