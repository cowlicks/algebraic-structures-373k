\documentclass[12pt,leqno]{article}

\usepackage{amsmath,amsfonts,amssymb,amscd,amsthm,amsbsy,upref}


\textheight=8.5truein
\textwidth=6.0truein
\hoffset=-.5truein
\voffset=-.5truein
\numberwithin{equation}{section}
\pagestyle{headings}
\footskip=36pt


\swapnumbers
\newtheorem{thm}{Theorem}[section]

\newtheorem{hthm}[thm]{*Theorem}
\newtheorem{lem}[thm]{Lemma}
\newtheorem{cor}[thm]{Corollary}

\newtheorem{prop}[thm]{Proposition}
\newtheorem{con}[thm]{Conjecture}
\newtheorem*{exer}{Exercise}
\newtheorem{bpe}[thm]{Blank Paper Exercise}
\newtheorem{apex}[thm]{Applications Exercise}
\newtheorem{ques}[thm]{Question}
\newtheorem{scho}[thm]{Scholium}
\newtheorem*{Exthm}{Example Theorem}
\newtheorem*{Thm}{Theorem}
\newtheorem*{Con}{Conjecture}
\newtheorem*{Axiom}{Axiom}



\theoremstyle{definition}
\newtheorem*{Ex}{Example}
\newtheorem*{Def}{Definition}


\newcommand{\lcm}{\operatorname{lcm}}
\newcommand{\ord}{\operatorname{ord}}
\def\pfrac#1#2{{\left(\frac{#1}{#2}\right)}}


\makeindex

\usepackage[parfill]{parskip}
\usepackage[singlespacing]{setspace}
\usepackage{listings}
\begin{document}




\thispagestyle{plain}
\begin{flushright}
\large{\textbf{Blake Griffith \\}
\textbf{HW 2:} Chapter 2, 5.2, 5.4, 6.2, 6.4, 6.7, 6.10, 7.2, 7.4, 8.3, 8.4, 8.5, 8.6  \& \\
Pre-Lecture Problems: Chapter 2, 5.1, 6.1, 7.1, 8.1\\
M373K \\
%Type your collaborators here. If there isn't any, then delete this line\\
Due 09/19, 2013 \\}
\end{flushright}

%\maketitle
\markboth{}{} \setcounter{section}{0} \baselineskip=18pt

\setcounter{tocdepth}{4}
%%%%%%%% Write the exercise number inside the brackets.


%%%%%%%%%%%%5.2%%%%%%%%%%%%%%%%%%%%%%%%%%%%%%%%%%%%%%%%%%%%%%%%%%%%%
\begin{exer}[5.2]
Prove that the intersection $K \cap H $ of subgroups of a group $G$ is a
subgroup of $H$, and that if $K$ is a normal subgroup of $G$, then $K \cap H$
is a normal subgroup of $H$.  
\end{exer}

\begin{proof}[Proof]
    
\item[(a)] Show $K \cap H$ is a subgroup of $H$. Where $K$ and $H$ are
    subgroups of $G$:

    $K \cap H$ is contained in $H$ because it only contains items
    which are in both $K$ and $H$.

    We have closure. Consider $a$ and $b$ in $K \cap H$. Then $ab$ is in $K$
    because $a$ and $b$ are in $K$ and $K$ is a group. Likewise $ab$ is in
    $H$. So since $ab$ is in both $K$ and $H$, we know $ab$ is in $K \cap H$.

    We have inverses. Again considering $a$, $a^{-1}$ must be in $K$ and $H$
    since they both contain $a$ and are groups. So $a^{-1} \in K \cap H$. 

    We have the identity, $I$. $K$ and $H$ are both groups so they must contain
    $I$. So $K \cap H$ must contain I.

\item[(b)] Show that if $K$ is a normal subgroup of $G$, then $K \cap H$ is a
    normal subgroup of $H$:

    By part (a), we know $K \cap H$ is a subgroup of $H$. So we must simply
    show it is normal to $H$.  Recalling the definition of a normal subgroup we
    must have for every $n$ in $K \cap H$ and every $h$ in $H$, $hnh^{-1}$ is
    in $K \cap H$.

    To see this, we consider $hnh^{-1} = c$. We know $c$ must be in $K$, since
    $K$ is a normal subgroup of $G$ and $h$ is in $G$ and $n$ is in $K$. 
    
    We also know that $c$ must be in $H$. Since $h$ is in $H$, and $n$ is $K
    \cap H$ and therefore in $H$. So $n$, $h$, and $h^{-1}$) are in the group
    $H$, so the result must be in $H$.

    So $c$ must be in $H$ and $K$, therefore it is in $K \cap H$. Since
    $hnh^{-1}$ is in $K \cap H$ it is a normal subgroup.
    
\end{proof}


%%%%%%%%%%%%5.4%%%%%%%%%%%%%%%%%%%%%%%%%%%%%%%%%%%%%%%%%%%%%%%%%%%%%
\begin{exer}[5.4]
Let $f:\mathbb R^{+} \rightarrow \mathbb C^{\times}$ be the map $f(x)=e^{ix}$.
Prove that $f$ is a homomorphism and determine its kernel and image.
\end{exer}

\begin{proof}[Proof]

    Let $a$ and $b$ be elements of $\mathbb R^+$ then we have
    $$
    f(a + b) = e^{i(a + b)}
    f(a)f(b) = e^{ia} e^{ib} = e^{i(a + b)}
    $$
    So $f$ is a homomorphism since $f(a + b) = f(a)f(b)$.

    To find the kernel of $f$ we consider the identity of $\mathbb C^{\times}$
    which is $1$.  Using this and Euler's equation to solve we see $e^{ix} =
    \cos(x) + i\sin(x) = 1$. Which requires $x = n 2 \pi$ for some integer $n$.
    So our kernel is

    $$
    \ker f = K = \left\{x \vert x = n 2 \pi \text{ for any integer } n \right\}
    $$

    The image of $f$ is given by $f(x) = e^{ix} $ for all $x$ in $\mathbb R^+$.
    This is simply a circle in the complex plane of radius one, centered at the
    origin. We can write this as 
    
    $$
    \text{im} f = f(\mathbb R^+) \left\{z \vert 1 = |z| \text{ for any complex number }z \right\}
    $$

\end{proof}

%%%%%%%%%%%%6.2%%%%%%%%%%%%%%%%%%%%%%%%%%%%%%%%%%%%%%%%%%%%%%%%%%%%%
\begin{exer}[6.2]
Describe all homomorphisms $\phi: \mathbb Z^{+} \rightarrow \mathbb Z^+$.
Determine which are injective, which are surjective and which are isomorphisms.
\end{exer}

\begin{proof}[Proof]

    For any $\phi$, the image of $\phi$ must be either $\mathbb Z$ or $n \mathbb Z$.
    Since these are the only subgroups of $\mathbb Z^+$.

    Any $\phi$ can be described by $\phi(1)$ since for any $a$ in $\mathbb Z$
    we can write $\phi(a) = \phi(\underbrace{1 + \dots + 1}_{a \text{ times}})
    = \underbrace{\phi(1) + \dots + \phi(1)}_{a \text{ times}} = a \phi(1)$.

    Each $\phi(1)$ gives a distinct homomorphism. In the case $\phi(1) \neq 0 \text{ or } 1$, $\phi$ is
    injective but not surjective. Since nothing maps to $1$. But each element of $\mathbb Z$ maps uniquely.

    If $\phi(1) = 0$, $\phi$ is neither surjective (nothing $\mapsto 1$) nor injective (everything $\mapsto 0$).

    If $\phi(1) = 1$, $\phi$ is the identity map so it is bijective (injective and surjective). 

\end{proof}


%%%%%%%%%%%%6.4%%%%%%%%%%%%%%%%%%%%%%%%%%%%%%%%%%%%%%%%%%%%%%%%%%%%%
\begin{exer}[6.4]Prove that in a group, the products $ab$ and $ba$ are conjugate elements.
\end{exer}

\begin{proof}[Proof]

    For $a$, $b$ in group $G$. The products $ab$ and $ba$ are conjugate
    elements if there exists an element $g$ in $G$ such that $ab = gbag^{-1}$.
    This is satisfied if we take $g = b^{-1}$. From this it follows that $ab =
    gbag^{-1} = b^{-1}bab = ab$ So $ab$ and $ba$ are conjugate elements.

\end{proof}


%%%%%%%%%%%%6.7%%%%%%%%%%%%%%%%%%%%%%%%%%%%%%%%%%%%%%%%%%%%%%%%%%%%%
\begin{exer}[6.7]
Let $H$ be a subgroup of $G$, and let $g$ be a fixed element of $G$. The
\emph{conjugate subgroup} $gHg^{-1}$ is defined to be the set of all conjugates
$ghg^{-1}$, with $h$ in $H$. Prove that $gHg^{-1}$ is a subgroup of $G$. 
\end{exer}

\begin{proof}[Proof]

    First, $gHg^{-1}$ is closed in $G$. We see this is we take an arbitrary $h$
    in $H$ then $ghg^{-1}$ is in $G$ since it is the product of elements in $G$
    which must be closed since it is a group.

    Next, $gHg^{-1}$ contains an inverse for each of its elements. Again
    consider $h$, and $g$, and let $a$ be some arbitrary element in the
    conjugate subgroup, $a = ghg^{-1}$. Then $h^{-1}$ must exist in $H$ since
    it is a group, and it must map to some element in the conjugate subgroup,
    let it be $b = gh^{-1}g^{-1}$. Then $a$ and $b$ are inverse elements since
    $ab = ghg^{-1}gh^{-1}g^{-1} = ghh^{-1}g^{-1} = gg^{-1} = 1$ and $ba =
    gh^{-1}g^{-1}ghg^{-1} = gh^{-1}hg^{-1} = gg^{-1} = 1$. So the conjugate
    subgroup has its inverses.

    Finally, $gHg^{-1}$ contains the identity. Since $H$ has an identity, then
    $gHg^{-1}$ has the element $g1g^{-1} = gg^{-1} = 1$. 

    So $gHg^{-1}$ is a subgroup of $G$.

\end{proof}


%%%%%%%%%%%%6.10%%%%%%%%%%%%%%%%%%%%%%%%%%%%%%%%%%%%%%%%%%%%%%%%%%%%%
\begin{exer}[6.10]Find all automorphisms of (a) a cyclic group of order $10$.
    (b) the symmetric group $S_3$.

\end{exer}

\begin{proof}[Proof]

    For $\phi$ to be a automorphism we must have $\phi : G \rightarrow G$ and $\phi(ab) = \phi(a) \phi(b)$.
    This implies that for some $x$ in $\phi$, $\vert x \vert = \vert \phi(x) \vert$. Otherwise, $\phi$ would
    not be a homomorphism. 

\item[(a)]

    Let $G$ be the cyclic group with order $\vert G \vert = 10$ and it is generated by
    some element $x$. So  $G = {1, x, x^2, \dots x^9}$. This group is completely determined
    by the map of $\phi(x)$, since $\phi(x^n) = \phi(x)^n$. 

    We know $\phi$ must map elements to the same order. For this to be the case $\phi(x)$ must map to
    elements that are coprime with the order. These are $x$, $x^3$, $x^7$, $x^9$. So there 
    are a total of 4 possible automorphisms.

\item[(b)]

    First we note that $\phi$ must send elements to the same order. Grouping
    the elemnts of $S_3$ by order gives. Order 1 is $1$. Order 2 is $y$, $xy$, $x^2y$. 
    Order 3 is $x$ and $x^2$. 
    If we alter the representation of the book of $S_3$ it is clear that 
    elements of order 3 can be generated by order 2.
    $$
    \begin{array}{ccc}
        1 & \rightarrow & 1 \\
        y & \rightarrow & a \\
        xy & \rightarrow & b \\
        x^2y & \rightarrow & c \\
        x & \rightarrow & ba    \\
        x^2 & \rightarrow & ca
    \end{array}
    $$
    Since order 3 elements $ba$ and $ca$ are determined by order 2 elements
    determining $\phi(a)$, $\phi(b)$, $\phi(c)$ determines $\phi(ba)$ and $\phi(ca)$.
    There are 3 order 2 elements so there are $3! = 6$ different ways to map them. 
    So we have 6 automorphisms.


\end{proof}


%%%%%%%%%%%%7.2%%%%%%%%%%%%%%%%%%%%%%%%%%%%%%%%%%%%%%%%%%%%%%%%%%%%%
\begin{exer}[7.2]An equivalence relation on $S$ is determined by the subset $R$
    of $S \times S$ consisting of those pairs $(a,b)$ such that $a \sim b$.
    Write axioms for an equivalence relation in terms of the subset $R$.

\end{exer}

\begin{proof}[Proof]

\item[Transitive] If $(a, b)$ and $(b, c)$ are in $R$, then $(a, c)$ is in $R$.

\item[Symmetric] If $(a, b)$ is in $R$, then so is $(b, a)$.

\item[Reflexive] If any pair of $R$ contains $a$, then $R$ contains $(a, a)$.

\end{proof}


%%%%%%%%%%%%7.4%%%%%%%%%%%%%%%%%%%%%%%%%%%%%%%%%%%%%%%%%%%%%%%%%%%%%
\begin{exer}[7.4] 
A relation $R$ on the set of real numbers can be thought of a subset of the
$(x,y)$-plane. With the notation of Exercise 7.2, explain the geometric meaning
of the reflexive and symmetric properties.

\end{exer}

\begin{proof}[Proof]

\item[Reflexive] This implies that for every set of coordinates $(x, y)$ in $R$
    the corresponding points $(x, x)$ and $(y, y)$ are also in $R$. And they
    lie on the line $x = y$.

\item[Symmetric] This implies that for every set of coordinates $(x, y)$ in
    $R$. There is a corresponding point $(y, x)$ which is also in $R$. This
    point is a reflection of $(x, y)$ across the line $x = y$. 

\end{proof}


%%%%%%%%%%%%8.3%%%%%%%%%%%%%%%%%%%%%%%%%%%%%%%%%%%%%%%%%%%%%%%%%%%%%
\begin{exer}[8.3] Does every group whose order is a power of a prime $p$
    contain an element of order $p$?

\end{exer}

\begin{proof}[Proof]

    Yes. Suppose the group $G$ is order $\vert G \vert = p^n$ where $p$ is a prime
    and $n$ is an integer.

    If $G$ contains some element $a$ then $\vert a \vert$ must divide $p^n$.
    By Lagrange's theorem. 

    The subgroup generated by $a$ must be order $p$, $p^2$, $p^3$, ..., $p^{n}$.
    If it is order $p$ we are done. Otherwise, $a$ is order $p^i$ where $2 \le
    i \le n$. If $<a>$ had no subgroups then every element in it would generate
    $<a>$. But this is not the case since the order of $a$ is $p^i$ you could
    take any power of $a$ less than $p^i$, which is not coprime to $p^i$, say
    $a^{p^{j}}$, and generate another subgroup which is smaller than $<a>$. 
    So every group that is not of prim order has a subgroup. You could continue
    reducing the size of your subgroups generators by choosing an element whos order is coprime
    with the order untill you reach an element that is order $p$. 

\end{proof}

%%%%%%%%%%%%8.4%%%%%%%%%%%%%%%%%%%%%%%%%%%%%%%%%%%%%%%%%%%%%%%%%%%%%
\begin{exer}[8.4] Does every group of order $35$ contain an element of order
    $5$? of order $7$? 

\end{exer}

\begin{proof}[Proof]

    In this group $G$, consider some element $a$ that is not the identity so
    $\vert a \vert \neq 1$. Let $x = \vert a \vert$. $x$ must divide the order
    of $G$ (see last problem) so it can be $5$, $7$, or $35$.

    If $G$ contains an element $a$ of order $35$. Then $a^7$ and $a^5$ are in
    the group, and $(a^7)^5 = (a^5)^7 = a^{35} = 1$ so there are elements of
    order $5$ and $7$ in the group.

    If $G$ does not contain an element of order $35$. Then it must contain
    elements of order $7$ and/or $5$.

    If we assume there are only elements of order $5$. Then each generator
    produces $4$ unique elements.  But there must be several ($n$) of these
    subgroups to fill $G$. But the number of elements produced by these $n$
    subgroups is $1 + 4n$ (the $+1$ is from the identity). But this cannot
    equal $35$ so we have a contradiction.

    Likewise, for a group of order $7$ elements, we would need $\vert G \vert =
    1 + 6n$ but this is a contradiction. 

    So $G$ must have bothe order $7$ and order $5$ elements.

\end{proof}

%%%%%%%%%%%%8.5%%%%%%%%%%%%%%%%%%%%%%%%%%%%%%%%%%%%%%%%%%%%%%%%%%%%%
\begin{exer}[8.5] A finite group contains an element $x$ of order $10$ and also
    an element $y$ of order $6$. What can be said about the order of $G$? 

\end{exer}

\begin{proof}[Proof]

    By the counting theorem the least it can be is order 30. Since 30 is the
    LCM of 6 and 10.

\end{proof}

%%%%%%%%%%%%8.6%%%%%%%%%%%%%%%%%%%%%%%%%%%%%%%%%%%%%%%%%%%%%%%%%%%%%
\begin{exer}[8.6] Let $\phi:G \rightarrow G'$ be a group homomorphism. Suppose
    that $|G|=18, |G'|=15$ and that $\phi$ is not the trivial homomorphism.
    What is the order of the kernel?

\end{exer}

\begin{proof}[Proof]
    
    Since ker $\phi$ is a subgroup in $G$, its order must divide $\vert G \vert = 18$ so its 
    order can be 1, 2, 3, 6, or 18. 

    Since im $\phi$ is a subgroup in $G'$, its order must divide $\vert G' \vert = 15$ so its
    order can be 1, 3, 5, or 15.

    We also know that $\lbrack G : \ker \phi] = \vert \text{im } \phi \vert$.
    And the counting formula applied here is $\vert G \vert = \vert \text{ker }
    \phi \vert \lbrack G : \text{ker } \phi \rbrack$.  Combining these we see
    $\vert G \vert = \vert \text{ker } \phi \vert \vert \text{im } \phi \vert$.
    This constraint gives us the solution: $\vert \text{ker } \phi \vert = 3
    \text{ and } \vert \text{im } \phi \vert = 5$.

\end{proof}

\section*{Pre-Lecture Problems}
%%%%%%%%%%%%Pre-Lecture Problem 5.1 %%%%%%%%%%%%%%%%%%%%%%%%%%%
\begin{exer}[5.1]  Let $\phi:G \rightarrow G'$ be a surjective group homomorphism. Prove that if $G$ is cyclic then $G'$ is cyclic. If $G$ is abelian then $G'$ is abelian.
\end{exer}

\begin{proof}[Proof]
\item[(a)] Show $G'$ is cyclic if $G$ is cyclic.

    Choose some $b$ in $G'$, since $\phi$ is surjective there exists some $c$
    in $G$ such that $\phi(c) = b$. Since $G$ is cyclic we can write $c = x^n$
    where $<x> = G$ So we write
    
    $$
    b = \phi(c) = \phi(x^n) = \phi(\underbrace{x \dots x}_{n \text{ times}})
    = \underbrace{\phi(x) \dots \phi(x)}_{n \text{ times}} = \phi(x)^n
    $$

    So any element in $G'$ can be written as a power of $\phi(x)$. So $G'$ is
    the cyclic group $<\phi(x)>$.

\item[(b)] Show $G'$ is Abelian if $G$ is Abelian.

    Let $a$ and $b$ be elements of $G'$. Then since $\phi$ is surjective $a$
    and $b$ correspond to some $c$ and $d$ in $G$ where $a = \phi(c)$, $b =
    \phi(d)$. Knowing this and the fact that $G$ is Abelian, we can write

    $$
    ab = \phi(c) \phi(d) = \phi(cd) = \phi(dc) = \phi(d) \phi(c) = ba
    $$

    So $ab = ba$, so $G'$ is Abelian too.

\end{proof}

%%%%%%%%%%%%6.1%%%%%%%%%%%%%%%%%%%%%%%%%%%%%%%%%%%%%%%%%%%%%%%%%%%%%
\begin{exer}[6.1]
Let $G'$ be a group of real matrices of the form $\left (  \begin{array}{cc}
 1 & x \\ 
   & 1 \end{array} \right )$. Is the map $\mathbb R^+ \rightarrow G'$ that sends $x$ to this  matrix an isomorphism?
\end{exer}

\begin{proof}[Proof]
    Let $a$ and $b$ be in $\mathbb R$. Then
    $$
    \phi(a + b) = 
    \left (  \begin{array}{cc}
        1 & a + b \\
        0 & 1 
    \end{array} \right )\\
    \phi(a) \phi(b)
    =
    \left (  \begin{array}{cc}
        1 & a \\
        0 & 1 
    \end{array} \right )
    \left (  \begin{array}{cc}
        1 & b \\
        0 & 1 
    \end{array} \right )
    =
    \left (  \begin{array}{cc}
        1 & a + b \\
        0 & 1 
    \end{array} \right ) \\
    $$

    So $\phi$ is a homomorphism since $\phi(a + b) = \phi(a) \phi(b)$.  To be
    an isomorphism, $\phi$ should be injective to its image.  This is true if
    $\ker \phi  = \{ I_{\mathbb R^+} \}$. The identity element of $\mathbb R^+$
    is $0$. And clearly

    $$
    \phi(0) = 
    \left (  \begin{array}{cc}
        1 & 0 \\
        0 & 1 
    \end{array} \right ) \\
    $$

    But we must check that no other elements are in the kernel. To show this we
    assume there is another element, $a$ in the kernel. So 
     
    $$
    \phi(a) = 
    \left (  \begin{array}{cc}
        1 & 0 \\
        0 & 1 
    \end{array} \right ) \\
    $$

    But this implies $a = 0$. So $\ker \phi = \{0\}$. This implies 
    that $\phi$ is injective. So $\phi$ is isomorphic to it's image.

\end{proof}


%%%%%%%%%%%%7.1%%%%%%%%%%%%%%%%%%%%%%%%%%%%%%%%%%%%%%%%%%%%%%%%%%%%%
\begin{exer}[7.1]
Let $G$ be a group. Prove that the relation $a \sim b$ if $b=gag^{-1}$ for some $g$ in $G$ is an equivalence relation.
\end{exer}

\begin{proof}[Proof]

    The relation is transitive. If $a \sim b$ and $b \sim c$. Then for some $g$
    and $g'$ in $G$ we have $b = gag^{-1}$ and $c = g'bg'^{-1} \implies b =
    g'^{-1}cg'$. Combining these gives $g'^{-1}cg' = gag^{-1} \implies c =
    (g'g)a(g^{-1}g'^{-1})$. But $g'g$ and $g'^{-1}g^{-1}$ are elements of $G$
    since they are products of elements of $G$. So $a \sim c$.

    The relation is symmetric. If $a \sim b$ then $b = gag^{-1}$ or $g^{-1}bg =
    a$ but $g^{-1}$ and $g$ are inverses and in $G$. So $b \sim a$.

    The relation is reflexive. If $a \sim a$ then $a = gag^{-1}$. Which is true

    So the relation is an equivalence relation.

\end{proof}


%%%%%%%%%%%%8.1%%%%%%%%%%%%%%%%%%%%%%%%%%%%%%%%%%%%%%%%%%%%%%%%%%%%%
\begin{exer}[8.1]
Let $H$be the cyclic subgroup of the alternating group $A_4$ generated by the permutation $(1 2 3)$. Exhibit the left and the right cosets of $H$ explicitly.
\end{exer}

\begin{proof}[Proof]

    I used Python and the SymPy package to do this. Info about sympy is at
    sympy.org.  The code is attached. The output is below. Note that
    the Permutations are indexed from zero, so \texttt{Permutation(0, 1, 3)} is
    $(124)$. Each \texttt{PermutationGroup} is a set of permutations (not
    strictly a group). And \texttt{Permutation(3)} is just the identity
    permutation.

\begin{lstlisting}
left cosets
PermutationGroup([
    Permutation(3),
    Permutation(3)(0, 1, 2),
    Permutation(3)(0, 2, 1)])
PermutationGroup([
    Permutation(1, 2, 3),
    Permutation(0, 1)(2, 3),
    Permutation(0, 2, 3)])
PermutationGroup([
    Permutation(1, 3, 2),
    Permutation(0, 1, 3),
    Permutation(0, 2)(1, 3)])
PermutationGroup([
    Permutation(0, 3, 1),
    Permutation(0, 3, 2),
    Permutation(0, 3)(1, 2)])
right cosets
PermutationGroup([
    Permutation(3),
    Permutation(3)(0, 1, 2),
    Permutation(3)(0, 2, 1)])
PermutationGroup([
    Permutation(1, 2, 3),
    Permutation(0, 2)(1, 3),
    Permutation(0, 3, 1)])
PermutationGroup([
    Permutation(1, 3, 2),
    Permutation(0, 3, 2),
    Permutation(0, 1)(2, 3)])
PermutationGroup([
    Permutation(0, 1, 3),
    Permutation(0, 3)(1, 2),
    Permutation(0, 2, 3)])
\end{lstlisting}


\end{proof}


\end{document}
