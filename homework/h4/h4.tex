\documentclass[12pt]{article}
\usepackage{amsfonts, amsmath, amsthm}

\setlength{\parskip}{1ex}
\setlength{\parindent}{0pt}

\newtheorem*{exer}{Exercise}

\newcommand{\img}{\text{img }}
\newcommand{\lcm}{\text{lcm }}
\newcommand{\aut}{\text{Aut }}
\newcommand{\cycle}[1]{(\mathbf{#1})}

\begin{document}

\textbf{Homework 4 -- Algebraic Structures} \\

\hrule

\begin{minipage}{.80\linewidth}
    \flushleft
    Ch 6: 7.10, 8.3, 8.4, M.2, M.3; Ch 7: 1.2, 2.3, 2.4, 2.7, 2.8, 2.9,
    2.13, 2.14, 2.17 \\
    Pre-lect: Ch 6: 7.1, 8.2; Ch 7: 1.1, 2.2 \\
\end{minipage}
\begin{minipage}{.20\linewidth}
    \flushright
    Blake Griffith
\end{minipage}

\begin{exer}[6.7.10]
    \begin{enumerate}
        \item Describe the orbit and the stabilizer of the matrix
            $\left[
                \begin{array}{cc}
                    1 & 0 \\
                    0 & 2 
                \end{array}
            \right]$
            under conjugation in the general linear group $GL_N (
            \mathbb{R})$.

        \item Interpreting the matrix in $GL_2(\mathbb{F}_5)$, find
            the order of the orbit.
    \end{enumerate}
\end{exer}

\begin{proof}
    We find the stabilizer of the matrix by solving the equation:
    \[
        \frac{1}{ad - bc} 
        \left[
            \begin{array}{cc}
                a & b \\
                c & d
            \end{array}
        \right]
        \left[
            \begin{array}{cc}
                1 & 0 \\
                0 & 2
            \end{array}
        \right]
        \left[
            \begin{array}{cc}
                d &-b \\
               -c & a
            \end{array}
        \right]
        =
        \left[
            \begin{array}{cc}
                1 & 0 \\
                0 & 2
            \end{array}
        \right]
    \]
    
    With the constraint $ad - bc \neq 0$. We find for $ad = 1$ the
    stabilizer is 
    \[
        \left[
            \begin{array}{cc}
                a & 0 \\
                0 & d
            \end{array}
        \right]
    \]

    For the orbit we compute
    \[
        \frac{1}{ad - bc} 
        \left[
            \begin{array}{cc}
                a & b \\
                c & d
            \end{array}
        \right]
        \left[
            \begin{array}{cc}
                1 & 0 \\
                0 & 2
            \end{array}
        \right]
        \left[
            \begin{array}{cc}
                d &-b \\
               -c & a
            \end{array}
        \right]
    \]
    And see with the constraint $ad - bc \neq 0$ the orbit is
    \[
        \frac{1}{ad - bc}
        \left[
            \begin{array}{cc}
                ad - 2bc & ab \\
                -cd & -bc + 2ad
            \end{array}
        \right]
    \]

\end{proof}

% % % % % % % % % % % % % % % % % % % % % % %

\begin{exer}[6.8.3]
    Exhibit the bijective map (6.8.4) explicitly, when $G$ is the
    dihedral group $D_4$ and $S$ is the set of vertices of a square.
\end{exer}

\begin{proof}
    Let the representation be  $D_4 = \{r, l | r^4 = l^2 = 1, lr =
    r^3l\}$.  We take the index $1$ as our element $s \in S$. This has
    the stabilizer $H = \{lr, 1\}$. Then the cosets are $G/H = \{\{lr,
    1\}, \{l, r\}, \{rl, r^2\}, \{r^2l, r^3\}\}$. 

    The orbit of our element is the indices $O_s = \{1, 2, 3, 4\}$. Then
    there is a bijection
    \begin{align*}
        G/H \rightarrow O_s \\
        \{lr, 1\} \mapsto 1 \\
        \{l, r\} \mapsto 2  \\
        \{rl, r^2\} \mapsto 3   \\
        \{r^2l, r^3\} \mapsto 4
    \end{align*}

\end{proof}

% % % % % % % % % % % % % % % % % % % % % % %

\begin{exer}[6.8.4]
    Let $H$ be the stabilizer of the index $1$ for the operation of the
    symmetric group $G = S_n$ on the set of indices $\{1, \dots, n\}$.
    Describe the left cosets of $H$ in $G$ and the map (6.8.4) in this
    case.
\end{exer}

\begin{proof}

    Let $x$ be the permutation which cycles the group $x = (\mathbf{123
    \dots n})$. And let the orbit of the first index be $O_1 = \{1, 2, 3
    \dots n\}$. Since each permutation in $H$ sends $1 \mapsto 1$. Then
    there is a bijection, $f(i) = x^{i-1}H$. Which sends index 1 to
    index i. More explicitly.
    \begin{align*}
        S_n / H \rightarrow O_1 \\
        H \mapsto 1 \\
        xH \mapsto 2 \\
        x^2 H \mapsto 3 \\
        \dots   \\
        x^{n-1} \mapsto n   \\
    \end{align*}
\end{proof}

% % % % % % % % % % % % % % % % % % % % % % %

\begin{exer}[6.M.2]
    \begin{enumerate}
        \item Prove that the set $\aut{G}$ of automorphisms of a group $G$
            forms a group, the law of composition being the composition
            of functions.
        \item Prove that the map $\phi : G \rightarrow \aut{G}$
            defined by $g \leadsto$ (conjugation by $g$) is a
            homomorphism, and determine its kernel.
        \item The automorphisms that are obtained as conjugation by a
            group element are called inner automorphisms. Prove that the
            set of inner automorphisms, the image of $\phi$, is a normal
            subgroup of the group Aut $G$.
    \end{enumerate}
\end{exer}

\begin{proof}
    \begin{enumerate}
        \item Closure is satisfied. Let $f$ and $g$ be in $\aut{G}$.
            Then $f \circ g$ is in $\aut{G}$ since $f \circ g(G) =
            f(g(G)) = f(G) = G$.

            The identity is in $G$. This is the identity homomorphism,
            which is also a automorphism.

            Inverses are satisfied. For any element $f$ in $\aut{G}$.
            $f$'s inverse exists since $f$ is a bijection. And since $f:
            G \rightarrow G$, $f^{-1}:G \rightarrow G$ too. So $f^{-1}$
            is an automorphism. So $f^{-1}$ is in $\aut{G}$.

        \item  For an $f$ in $G$ we define the homomorphism as $\phi(f)
            = \psi_f$. Where $\psi_f(x) = fxf^{-1} \; \forall x \in G$.
            So for any $f, g$ in $G$ we have $\phi(fg) = \psi_{fg}$ and
            $\psi_{fg}(x) = fgxg^{-1}f^{-1} = \psi_f(gxg^{-1}) =
            \psi_f( \psi_g(x)) = \psi_f \circ \psi_g(x) \implies \psi_f
            \psi_g = \phi(f)\phi(g)$. So $\phi$ is a homomorphism.

            The kernel $K$ of this homomorphism must map to the identity
            homomorphism, for some element $k$ in the kernel  $\phi(K) =
            \psi_K \implies psi_k(x) = kxk^{-1} = 1$. This is true for
            all elements that commute with every element of the group.
            So the kernel is the center of G.

        \item Let $c$ be any element in the center of $G$ so $\psi_c$ is
            an inner automorphism, and let $\phi$ be an arbitrary
            automorphism.  So $\phi \psi_c \phi^{-1} \implies \phi(x)
            \psi_c(x) \phi(x)^{-1}$. Since $\phi$ is an automorphism
            $\phi(x)$ is in $G$ and so is it's inverse. Let $\phi(x) =
            a$ then we have $a\psi_c(x)a^{-1} = acxc^{-1}a^{-1} =
            \psi_{ca}(x)$, which is in $\aut{G}$ so the inner
            automorphisms are a normal subgroup.

    \end{enumerate}

\end{proof}

% % % % % % % % % % % % % % % % % % % % % % %

\begin{exer}[6.M.3]
    Determine the groups of automorphisms (see Exercise M.2) of the
    group \textbf{(1)} $C_4$ \textbf{(2)} $C_6$, \textbf{(3)} $C_2
    \times C_2$, \textbf{(4)} $D_4$, \textbf{(5)} the quaternion group
    $H$.
\end{exer}

\begin{proof}
    \begin{enumerate}
        \item Recall automorphisms of a cyclic group must send
            generators to generators, so we have the group of
            homomorphisms $x \mapsto \{x, x^3\}$.

        \item By the same logic as part 1 we have $ x \mapsto \{x, x^5\}$.

        \item By guessing we have $(x, y) \mapsto \{(x,y), (y,x)\}$
            
        \item Again by guessing $r \mapsto r, l \mapsto \{l, l^3\}$.

        \item Still guessing, identity map, in cycle notation $(ijk),
            (ikj), (ij), (ik), (jk)$.

    \end{enumerate}

\end{proof}

% % % % % % % % % % % % % % % % % % % % % % %

\begin{exer}[7.1.2]
    Let $H$ be a subgroup of a group $G$. Describe the orbits for the
    operation of $H$ on $G$ by left multiplication.
\end{exer}

\begin{proof}
    If we choose some $h$ in $H$ then $O_h = H$ since $H$ is closed. If
    we choose some $g$ in $G$ then $O_g = G - H$. Because if $H$ sent
    $g$ to $H$ then for some $h_1$ and $h_2$ in $H$: $h_1
    g = h_2 \implies g = h_2 h_{2}^{-1}$ so $g$ would be in $H$ which is
    a contradiction.

\end{proof}

% % % % % % % % % % % % % % % % % % % % % % %

\begin{exer}[7.2.3]
    A group $G$ of order 12 contains a conjugacy class of order 4. Prove
    that the center of $G$ is trivial.
\end{exer}

\begin{proof}
    Since $|C_x| = 12$ and $|G| = 12$, $|Z(x)| = 3$. And since $|Z(x)|
    \neq |G|$, $x$ cannot be in the center of $G$. Therefore $|Z(x)|$
    must be greater than $|Z|$, since it must include the center and an
    additional element. So $|Z| < 3$, but $|Z|$ cannot be $2$ since $Z$
    is a subgroup of $Z(x)$ and its order must divide $Z(x)$. So it must
    be order 1. And therefore it must be the trivial group.

\end{proof}

% % % % % % % % % % % % % % % % % % % % % % %

\begin{exer}[7.2.4]
    Let $G$ be a group, and let $\phi$ be the \textbf{n}th power
    map:$\phi(x) = x^n$. What can be said about how $\phi$ acts on
    conjugacy classes?
\end{exer}

\begin{proof}
    Consider two conjugate elements $a$ and $b$ such that $a = gbg^{-1}$
    for some $g \in G$. Then $\phi$ acting on $a$ gives:
    \[
        \phi(a)= a^n = (gbg^{-1})^n =
        (gbg^{-1})(gbg^{-1})(gbg^{-1}) \dots 
    \]
    We can then cancel out the inner $gg^{-1}$ factors to get
    \[
        \phi(a) = g b^n g^{-1}
    \]
    So we can say it raises elements in the conjugacy classes to the
    power $n$.

\end{proof}

% % % % % % % % % % % % % % % % % % % % % % %

\begin{exer}[7.2.7]
    Rule out as many as you can, as class equations for a group of order
    10:
    \[
        1 + 1 + 1 + 2 + 5, \quad 1 + 2 + 2 + 5, \quad 1 + 2 + 3 + 4,
        \quad 1 + 1 + 2 + 2 + 2 + 2
    \]
\end{exer}

\begin{proof}
    \begin{enumerate}
        \item No. There are $3$ elements that have a conjugacy class of
            order $1$, so there are $3$ elements with a centralizer of
            order $10$. So there must be at least $3$ elements in the
            center.

            But there is a conjugacy class of order $5$. Which implies
            some element has a centralizer of order $2$. But the
            centralizer must contain the center. Since the center is
            order $3$ this is impossible.

        \item This seems fine\dots

        \item No. A conjugacy class of order $4$ is impossible because
            it would require a non-integer order of the centralizer.

        \item No. Since there are two elements with conjugacy classes of
            order $1$, there are two elements in the center.

            But there is also a conjugacy class of order $2$ which
            implies a centralizer of order $5$. However this
            centralizer cannot contain the center since the order of the
        center does not divide its order.

    \end{enumerate}

\end{proof}

% % % % % % % % % % % % % % % % % % % % % % %

\begin{exer}[7.2.8]
    Determine the possible class equations of nonabelian groups of order
    \textbf{(1)}8, \textbf{(2)}21.
\end{exer}

\begin{proof}
    \begin{enumerate}
        \item For a group of order 8 we note that the divisors are $1,
            2, 4, 8$. Then as always we have:
            \[
                1 \times 8
            \]

            Trying class equations with a center of order $2$ we find:
            \[
                1 + 1 + 2 + 2 + 2
            \]
            Since the center is order $2$ here the centralizer's must
            have order greater than $3$, so we can't use any conjugacy
            classes of order $4$.

        \item For a group of order $21$ we note the divisors are $1, 3,
            7, 21$. Then we have the trivial class equation:
            \[
                1 \times 21
            \]
            Trying groups with a center of order $1$ we see that there
            can be only one class equation that correctly sums to $21$.
            \[
                1 + 3 + 3 + 7 + 7
            \]
            All other attempts at forming class equations fail.
            
    \end{enumerate}

\end{proof}

% % % % % % % % % % % % % % % % % % % % % % %

\begin{exer}[7.2.9]
    Determine the class equations for the following groups: \textbf{(1)}
    the quaternion group, \textbf{(2)} $D_4$, \textbf{(3)} $D_5$,
    \textbf{(4)} the subgroup of $GL_2(\mathbb{F}_3)$ of invertible
    upper triangular matrices.
\end{exer}

\begin{proof}
    \begin{enumerate}
        \item The quaternions are order 8 and non-abelian. From the
            previous problem we know that the class equations that
            correspond to a group of order 8 are $1 \times 8$ and $1 + 1
            + 2 + 2 + 2$. Since the quaternions are not abelian, the
            center is not the whole group. So the class equation is the
            latter option.
        
        \item $D_4$ is also non-abelian, and order 8. So it has the same
            class equation as the quaternions. $1 + 1 + 2 + 2 + 2$.

        \item $D_5$ is order $10$ and non-abelian. Trying class
            equations with a center of order $1$ we see only $1 + 2 + 2
            + 5$ works out to the correct sum. We cannot have a center
            of any other order because it would not divide the
            corresponding centralizer's orders.

        \item ?
    \end{enumerate}

\end{proof}

% % % % % % % % % % % % % % % % % % % % % % %

\begin{exer}[7.2.13]
    Let $N$ be a normal subgroup of a group $G$. Suppose that $|N| = 5$
    and that $|G|$ is an odd integer. Prove that $N$ is contained in the
    center of $G$.
\end{exer}

\begin{proof}
    $N$ is cyclic since $|N|$ is prime. Note that normal subgroups are
    unions of conjugacy classes. To see this note that by a definition
    of a normal subgroup
    \[
        \forall g \in G, \forall n \in N: \; gng^{-1} \in N \implies
        \forall n \in N: \; C_n \subseteq N
    \]
    The conjugacy classes that compose $G$ and $N$ must be odd since
    $|G|$ is odd so the possible class equations of $N$ are:
    \[
        |N| = 1 + 1 + 3 \quad \text{or} \quad |N| = 1 \times 5
    \]
    In either case the center is at least order 2. So a non-identity
    element $x$ must exist in $N$ that commutes with the whole group.
    Since $N$ is cyclic, this $x$ must generate all of $N$. Since $Z$ is
    a group, and $x$ is in $Z$, $N$ is in $G$.

\end{proof}

% % % % % % % % % % % % % % % % % % % % % % %

\begin{exer}[7.2.14]
    The class equation of a group $G$ is $ 1 + 4 + 5 + 5 + 5$.
    \begin{enumerate}
        \item Does $G$ have a subgroup of order 5? If so, is it a normal
            subgroup?
        \item Does $G$ have a subgroup of order 4? If so, is it a normal
            subgroup?
    \end{enumerate}
\end{exer}

\begin{proof}
    \begin{enumerate}
        \item There exists a subgroup of order 5, this is because there
            is a conjugacy class of order 4 which implies a centralizer
            of order 5. 

            This subgroup is normal. It is order 5 so it must be cyclic,
            and therefore all its elements commute with each other.

        \item There exists a subgroup of order 4 because there is a
            conjugacy class of order 5 which implies a centralizer of
            order 4.

            The subgroup is not normal. Since any non identity element
            in it has a conjugacy class that is larger than the group.
            So the conjugates of all the elements in the group cannot be
            contained inside of it.

    \end{enumerate}

\end{proof}

% % % % % % % % % % % % % % % % % % % % % % %

\begin{exer}[7.2.17]
    Use the class equation to show that a group of order $pq$, with $p$
    and $q$ prime, contains an element of order $p$.
\end{exer}

\begin{proof}

    Note that the order of the conjugacy classes must divide the order
    of the group, and the groups divisors are $1, p, q$. So there are
    only 3 possible class equations.
    \[
        1 \times p + q \quad 1 \times q + p \quad 1 \times pq
        \text{ (cyclic)}
    \]
    Now recall that any group who's order is the product of two relatively
    prime integers $r$ and $s$ is isomorphic to the product of two
    cyclic groups of order $r$ and $s$. This implies the class equation:
    $1 \times pq$. So since the group is cyclic, it has cyclic subgroups
    that are the order of its divisors $p$, and $q$. Therefore it
    contains an element of order $p$.

    



\end{proof}

% % % % % % % % % % % % % % % % % % % % % % % % % % % % % % % % % %
% % % % % % % % % % % % % % % % % % % % % % % % % % % % % % % % % %
\section*{Pre-Lecture Problems}

\begin{exer}[6.7.1]
    Let $G = D_4$ be the dihedral group of symmetries of the square.
    \begin{enumerate}
        \item What is the stabilizer of a vertex? Of an edge?
        \item $G$ operates on the set of two elements consisting of the
            diagonal lines. What is the stabilizer of a diagonal?
    \end{enumerate}
\end{exer}

\begin{proof}
    Let the labels, going from the top left corner in the clockwise
    direction be $a, b, c, d$. And let $r$ be a clockwise rotation, and
    $l$ be a reflection across the vertical axis of symmetry. Then we
    have the group $\{r, l | r^4 = l^2 = 1\}$.

    \begin{enumerate}
        \item Then the stabilizer of vertices $a, c$ is $\{1, rl\}$. The
            stabilizer of vertices $b, d$ is $\{1, r^3l\}$. 

            The stabilizer of edges $ab$ and $cd$ is $\{1, l\}$. The
            stabilizer of edges $bc$ and $da$ is $\{1, r^2l\}$.

        \item The stabilizer of the diagonals $ac$ and $bd$ are $\{1,
            r^2, rl\}$.

    \end{enumerate}

\end{proof}

% % % % % % % % % % % % % % % % % % % % % % % % % % % % % % % % % %

\begin{exer}[6.8.2]
    What is the stabilizer of the coset $[aH]$ for the operation of $G$
    on $G/H$?
\end{exer}

\begin{proof}
    $G_{aH} = aHa^{-1}$.

    To see this, choose some $g \in aHa^{-1}$ then for some $h \in H$,
    we have $g = aha^{-1}$. So $gaH = aha^{-1}aH = ahH = aH$. So
    $aHa^{-1} \subseteq = G_{aH}$.

    Next take some $g \in G_{aH}$. Then for some $h, h', h'' \in H$, we
    have $gah = ah' \rightarrow g = ah'h^{-1}a^{-1} = ah''a^{-1}$. So
    $G_{aH} \subseteq aHa^{-1}$. Therefore $G_{aH} = aHa^{-1}$.

\end{proof}

% % % % % % % % % % % % % % % % % % % % % % % % % % % % % % % % % %

\begin{exer}[7.1.1]
    Does the rule $g * x = x g^{-1}$ define an operation of $G$ on $G$?
\end{exer}

\begin{proof}
    Yes. Checking the group operation axioms. $1 \times x = x1 = x$. And
    $fg \times x = f \times (xg^{-1}) = x g^{-1} f^{-1}$. So both axioms
    work, so it is a group operation.

\end{proof}

% % % % % % % % % % % % % % % % % % % % % % % % % % % % % % % % % %

\begin{exer}[7.2.2]
    A group of order 21 contains the conjugacy class $C(x)$ of order 3.
    What is the order of $x$ in the group?
\end{exer}

\begin{proof}
    With the given information we can determine $|Z(x)|=7$ since the
    order of the centralizer times the order of the conjugacy class must
    equal the order of the group.

    Since the centralizer is a subgroup, and prime order, it must be
    cyclic. Since $x$ is in the aforementioned group, it must be order
    $7$.

\end{proof}

\end{document}
