\documentclass[12pt]{article}
\usepackage{amsfonts, amsmath, amsthm}

\setlength{\parskip}{1ex}
\setlength{\parindent}{0pt}

\newtheorem*{exer}{Exercise}

\newcommand{\img}{\text{img }}
\newcommand{\lcm}{\text{lcm }}
\newcommand{\aut}{\text{Aut }}
\newcommand{\cycle}[1]{(\mathbf{#1})}

\begin{document}

\textbf{Homework 4 -- Algebraic Structures} \\

\hrule

\begin{minipage}{.80\linewidth}
    \flushleft
    Ch 7: 3.1, 3.3, 3.4, 4.7, 4.9, 5.3, 5.6, 5,7, 5.12, 6.4, 6.5 \\
    Pre-lect: 5.1 \\
\end{minipage}
\begin{minipage}{.20\linewidth}
    \flushright
    Blake Griffith
\end{minipage}

\begin{exer}[7.3.1]

    Prove the Fixed Point Theorem (7.3.2).

\end{exer}

\begin{proof}

    Recall the counting formula
    \[
        |G| = |G_s||O_s|
    \]
    and since the orbits partition the set
    \[
        |S| = |O_1| + \dots + |O_k|
    \]
    So for some element $s \in S$, $|O_s|$ must divide $|G|$, so it can be
    $p$ or $1$. $|O_s|$ must also divide $|S|$. But since $p$ does not
    divide $|S|$, $|O_s| = 1$. Then by the counting theorem $|G_s| =
    |G|$. So $s$ stabilizes the whole group.

\end{proof}

% % % % % % % % % % % % % % % % % % % % % % %

\begin{exer}[7.3.3]

    A non-abelian group $G$ has order $p^3$, where $p$ is prime.

    \begin{enumerate}
        \item What are the possible orders of the center $Z$?

        \item Let $x$ be an element of $G$ that isn't in $Z$. What is
            the order of its centralizer $Z(x)$?

        \item What are the possible class equations for $G$?
    \end{enumerate}

\end{exer}

\begin{proof}

    \begin{enumerate}
        \item First we can rule out the case $Z = 1$ using the class
            equation since it would imply $p^3 = 1 + (\text{\it{sum of
            $p$ and $p^2$}})$. Which is impossible. 

            We can rule out $p^3$ since it would imply the group is
            abelian.

            We can rule out $p^2$ since the centralizer for any element
            not in the center would have to be larger order than $p^2$
            and divide $p^3$, but not be equal to $p^2$.

            So we are left with $|Z| = p$.

        \item The centralizer must be larger than $Z$ since it must
            contain $x$, but it cannot be the same size of the group
            $p^3$. And its order must divide the order of the group so
            we are left with $p^2$.

        \item Applying the information above. We can only have $1 \times
            p + p \times (p^2 - 1)$.
            
    \end{enumerate}

\end{proof}

% % % % % % % % % % % % % % % % % % % % % % %

\begin{exer}[7.3.4]

    Classify groups of order 8.

\end{exer}

\begin{proof}

    For abelian groups we start with $C_8$, then by inspection we also
    have $C_4 \times C_2$, and $C_2 \times C_2 \times C_2$.

    For non-abelian groups, note that $8 = 2^3$. So we can apply part 3
    of the previous problem and note that the class equation must be $1
    + 1 + 2 + 2 + 2$. Also every non-identity element must be either
    order 2 or 4.

    But if every element were order 2 the group would be abelian. So
    there is at least one element $x$ of order 4. 

    There are two non-abelian groups of order 8. The dihedral group on a
    square, and the quaternions. But I'm not sure how to derive these
    with just the order of the group.

\end{proof}

% % % % % % % % % % % % % % % % % % % % % % %

\begin{exer}[7.4.7]

    Let $G$ be a group of order $n$ that operates non-trivially on a set
    of order $r$. Prove that if $n > r!$, then $G$ has a proper normal
    subgroup.

\end{exer}

\begin{proof}

    Since $G$ operates non-trivially we know for some $g \in G$ and $s
    \in S$ that $gs \neq 1$\dots

\end{proof}

% % % % % % % % % % % % % % % % % % % % % % % % % % % % % % % % % %

\begin{exer}[7.4.9]

    Let $x$ be an element of a group $G$, not the identity, whose
    centralizer $Z(x)$ has order $pq$, where $p$ and $q$ are primes.
    Prove that $Z(x)$ is abelian.

\end{exer}

\begin{proof}

    Suppose $Z(x)$ is not abellian. 

    $Z(x)$ is itself a group. Which must have a center containing $x$
    and $1$, so $|Z| > 1$. Since the center is a subgroup its order must
    divide the order of the group, so it is either $p, q, pq$. 

    If the order of $Z$ is $pq$, $Z(x)$ is abellian and we are done.

    With out loss of generality suppose the order is $p$, consider
    the centralizer of some element $y \in Z(x) - Z$. Its centralizer
    must have order greater than $p$ and it must divide $pq$, so it must
    be order $pq$. So it commutes with the whole group. 

    So all elements not in the center commute, and so does the center.
    But this is a contradiction, so $Z(x)$ must be abellian.

\end{proof}

% % % % % % % % % % % % % % % % % % % % % % % % % % % % % % % % % %

\begin{exer}[7.5.3]

    Determine the orders of the elements of the symmetric group $S_7$.

\end{exer}

\begin{proof}

    We have the obvious cases for $1$ through $7$ cycles which gives
    order $1, 2, 3, 4, 5, 6, 7$ elements. 

    Then we have the orders that arise from products of disjoint cycles.
    We start counting these down from 7 cycles noting that 7 and 6
    cycles cannot form disjoint products with any of the cycles in
    $S_7$. 
    
    With 5 cycles we can form products with disjoint 2
    cycles yielding an element of order 10.

    With 4 cycles we can form products with disjoint 2 and 3 cycles
    yielding elements of order 8, and 12.

    Counting lower results in double counting so our full list of the
    order of all elements of $S_7$ is $1, 2, 3, 4, 5, 6, 7, 8, 10, 12$.

\end{proof}

% % % % % % % % % % % % % % % % % % % % % % % % % % % % % % % % % %

\begin{exer}[7.5.6]

    Find all subgroups of $S_4$ of order 4, and decide which ones are
    normal.

\end{exer}

\begin{proof}

    We have the obvious cases, the subgroups generated by any of the 12
    (4!) four cycles. The canonical example of these is $\{1,
        \cycle{1234}, \cycle{13}\cycle{24}, \cycle{1432} \}$.

    Then by inspection we note that 2 disjoint transpositions generate
    subgroups of order 4. There 6 (4 choose 2) of these, the canonical
    example being $\{1, \cycle{12}, \cycle{34}, \cycle{12}\cycle{34}\}$. 

    So in total there are 18 subgroups of order 4 in $S_4$.

\end{proof}

% % % % % % % % % % % % % % % % % % % % % % % % % % % % % % % % % %

\begin{exer}[7.5.7]

    Prove that $A_n$ is the only subgroup of $S_n$ of index 2.

\end{exer}

\begin{proof}

    Suppose there is another subgroup $X$ of index 2. Then recall that
    any subgroup of index 2 is normal. So $X$ is normal.

    Then since $X$ is normal it must contain a 3-cycle (see the proof on
    page 202). 

    Since the group is normal and contains a 3-cycle, it must contain
    all 3-cycles since they form a conjugacy class. 

    So $X$ must contain $A_n$ since the 3-cycles generate $A_n$. But
    $|A_n| = |X|$ so the groups must be equal. But this is a
    contradiction. So $A_n$ must be the only subgroup of index 2.

\end{proof}

% % % % % % % % % % % % % % % % % % % % % % % % % % % % % % % % % %

\begin{exer}[7.5.12]

    Determine the class equations of $S_6$ and $A_6$.

\end{exer}

\begin{proof}

    Recall that cycles of the same shape are in the same conjugacy
    class. So we need to count each kind of cycle. Note that I skip
    cases which are double counting.

    For single 2-cycles there are 15 (6 choose 2).

    For products of two 2-cycles there are 45 (6 choose 2 * 4 choose 2 /
    2! for commuting cycles)

    For products of three 2-cycles, there are 30. (6 choose 2 * 4 choose
    2 / 3! for commuting cycles)

    Then there are the 3-cycles, 40 (6 choose 3 * 2! for each ordering)

    Then there products of 2-cycles and 3-cycles, 120 (6 choose 2 * 4
    choose 2 * 2!)

    For products of two 3-cycles, 40 (6 choose 3 * 3 choose 3 * 2! * 2!
    / 2!).

    For 4-cycles there are 90 (6 choose 4 * 3! for each ordering)

    For products of 4-cycles and 2-cycles there are 90.

    For 5-cycles there are 144 (6 choose 5 * 4! for each ordering)

    For 6-cycles there are 120 (5! for each ordering)

    So the class equation is:
    \[
        1 + 15 + 45 + 15 + 40 + 120 + 40 + 90 + 90 + 144 + 120 = 720
    \]

    For the alternating group we have all the even cycles from above.
    These are the products of two 2-cycles, the products of 4-cycles and
    2-cycles and all products of 3-cycles, and all 5-cycles. So we have:
    \[
        1 + 45 + 90 + 40 + 40 + 144 = 360
    \]

\end{proof}

% % % % % % % % % % % % % % % % % % % % % % % % % % % % % % % % % %

\begin{exer}[7.6.4]

    Let $H$ be a normal subgroup of prime order $p$ in a finite group
    $G$. Suppose that $p$ is the smallest prime that divides the order
    of $G$. Prove that $H$ is in the center $Z(G)$.

\end{exer}

\begin{proof}

    Since $H$ is normal it is a union of conjugacy classes. So there is
    some combination of terms in the class equation for $G$ that sum to
    $|H| = p$. Since $1 \in H$ and it corresponds to a $1$ in the class
    equation the rest of the terms in $H$ must sum to $p - 1$ in the
    class equation. But since these terms must also divide $|G|$, and
    are smaller than $p$ which is the smallest prime to divide $|G|$
    they must all be $1$, so they must all be in the center.

\end{proof}

% % % % % % % % % % % % % % % % % % % % % % % % % % % % % % % % % %

\begin{exer}[7.6.5]

    Let $p$ be a prime integer and let $G$ be a $p$-group. Let $H$ be a
    proper subgroup of $G$. Prove that the normalizer $N(H)$ of $H$ is
    strictly larger than $H$, and that $H$ is contained in a normal
    subgroup of index $p$.

\end{exer}

\begin{proof}

    \dots 

\end{proof}

% % % % % % % % % % % % % % % % % % % % % % % % % % % % % % % % % %
% % % % % % % % % % % % % % % % % % % % % % % % % % % % % % % % % %

\section*{Pre-Lecture Problems}

\begin{exer}[7.5.1]

    \begin{enumerate}
        \item Prove that the transpositions $\cycle{12}, \cycle{23},
            \dots \cycle{n - 1, n}$ generate the symmetric group $S_n$. 

        \item How many transpositions are needed to write the cycle
            $\cycle{123 \dots n}$?

        \item Prove that the cycles $\cycle{12 \dots n}$ and
            $\cycle{12}$ generate the symmetric group $S_n$.
    \end{enumerate}

\end{exer}

\begin{proof}

    \begin{enumerate}
        \item Let some arbitrary cycle be $\cycle{i_1, i_2, \dots i_j}$.
            Then we can generate this with the given transpositions.
            First we note that we can generate adjacent indices as
            follows
            \[
                \cycle{i_1, i_1 \pm 1} \cycle{i_1 \pm 1, i_1 \pm 2}
                \dots \cycle{i_1 \pm k_1, i_2} = \cycle{i_1, i_2}
            \]
            Similarly $\cycle{i_2, i_3} = \cycle{i_2, i_2 \pm 1}
            \dots \cycle{i_2 \pm k_2, i_3}$. So we can write
            $\cycle{i_1, i_2, i_3} = \cycle{i_1, i_2} \cycle{i_2, i_3}$.

            Generating the rest of the given cycle follows by induction.

        \item $n - 1$ by counting: 
            \[
                \cycle{123 \dots n} = \underset{1}{\cycle{12}}
                \underset{2}{\cycle{23}} \dots \underset{n-1}{\cycle{n-1,
                n}}
            \]

        \item Using the given cycles we can generate any adjacent
            transpositions since  $\cycle{i, i+1} = \cycle{123 \dots
            n}^{i+1} \cycle{12} \cycle{123 \dots n}^{1-i}$. Then
            applying the first part of this problem we can generate
            $S_n$.

    \end{enumerate}

\end{proof}

% % % % % % % % % % % % % % % % % % % % % % % % % % % % % % % % % %

\end{document}
